\documentclass[../m3y]{subfiles}
\externaldocument{../m3y}
\begin{document}
My stay in Le\thinspace{} Havre, though brief, was memorable in that V-E (Victory in Europe) Day occurred on the same day that we shipped out, on 4\&8 boxcars once again, en route to southern France. All of the men, I believe, were hospital dischargees returning to duty and, as we were marched down the back streets to the railhead, there was one heck of a celebration on the main street where the locally stationed GI's were marching to the cheers, hugs, and kisses of the delirious Frenchmen and women. Most of these were Black soldiers assigned to Quartermaster duty, which produced a lot of grousing from our group \ldots\ typical of the infantry, we thought \ldots\ we did all the heavy lifting and they get all the glory. Infantrymen like to gripe but, being at the bottom of the barrel when it came to duty, they often had quite a bit worthy of their gripes.

The train trip took well over a week to reach Marseille where we were dispersed to various assignments. As the war in the Pacific was still going on, the Army quickly set up three large staging areas, made up of endless rows of 16-man tents, these to accommodate the large number of troops that were to be redeployed from Marseille, through the Suez Canal finally being assembled at some island or other place to bring their force against the stubborn Japanese.

I took Typing I for a semester in High School and no single course has served me better. In this case it resulted in my assignment as a clerk-typist to the Troop Movement Office in Calas Staging Area, the largest of the three areas feeding the port, holding as many as 70,000 men, some with enough service points to be heading home, but most with still more combat experience to come. Our office was responsible for establishing the troops, usually a whole division at once, in their designated areas and then facilitating their subsequent boarding ship in Marseille. Most of my work consisted of typing up troop movement orders and answering the telephone. Tony Germano,who was to become a close friend, was assigned as a filing clerk. Two German POW's were also assigned to our office, one as a general assistant, the other as the operator of our mimeograph machine. The general assistant was a very dour German who devoted every free moment to the studying of English, while the other was a young man our own age, friendly, accommodating in every way, and very conscientious about doing his job. Many other POW's worked around the post, most of them unguarded until they returned to their barricaded area at night. They knew good duty when they saw it and didn't cause any trouble.

\begin{figure}
\centering
\includegraphics{calas-dining}\\
\medskip
{\newtimes\textsc{Al fresco dining, Calas, 1945 (Me on right)}}
\end{figure}

As long as we did our job reliably this was an easy assignment for all of us as well. Our tents were warmed with the usual GI stoves (southern France got cold in the winter months which followed) and, in that our officers were also ex-combat soldiers who knew what most of us had come through, there was almost a complete absence of typical Army ``chicken-\underline{\hspace{4em}}'' Other than the Army's monthly check for VD by the medical personnel (when we all had to assemble naked under our raincoats and helmet liners), we seldom had any parade or company inspections. It was here that I became close friends with Tony Germano, visiting him later in Chicago, Howard Samp, Bill Martin, Thomas Wammes, Charlie Monson, a Mormon, and Bill Shaw, who were enjoyable tent mates. Bill Shaw was by far the brightest, enjoying long discussions on most any subject that interested him, while Bill Martin, who became my roommate on our subsequent  assignment to Aix-en-Provence, was more than a little bright. Though Bill Martin, Charlie Monson and Bill Shaw already had some college experience, I found that I didn't lag too far behind when we got in our bull sessions, this experience helping to take out the mystique I associated with college training, encouraging me to formulate more ambitious educational plans than I had upon entering the service. Seeing that they didn't seem too much smarter than me was a learning experience in itself.
\end{document}
