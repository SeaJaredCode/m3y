\documentclass[../m3y]{subfiles}
\externaldocument{../m3y}
\begin{document}
It is not surprising that one's baptism in fire remains as a clear memory. In my case, our unit was assembled facing a hillside cemetery outside of Nothum (the town being identified on my subsequent visit with Allan in the 1970s), and we were told that we could expect to find the enemy someplace close by. We spread out in proper infantry fashion, going past the cemetery and continuing down the barren hill on the further side. There was a road at the base of the hill, with heavy woods coming to the edge of the road on the other side. All was quiet, with the bulk of our company working its way down the hillside when all hell broke out from the woods beyond. Wisely, the Germans had held their fire until most of us were exposed. In my case, I ran as fast and, to reduce my vulnerability, as erratically as I could toward the gully alongside the road at the bottom, and gained some protection as I snuggled up against the few feet of protection available.

What followed was quite tragic as many of the men, being untrained as infantrymen, immediately hit the ground upon hearing gunfire instead of running for the protection of the gully below. Upon realizing their mistake, they would get up and try to join the rest of us at the bottom but got mowed down by concentrated German fire. One recent transferee that I had gotten to know was a fellow from Belmont, married with one child. I was sickened to see him killed, having been one of the ones who made the wrong call.

\begin{figure}[h]
\centering
\includegraphics[width=0.8\textwidth]{cemetery-hill}\\
\medskip
{\newtimes\textsc{Cemetery Hill, Nothum, 2004}}
\end{figure}

This became a tough situation, with us remaining pinned down along the gully for a few days. Our immediate concern was our left flank, where the woods crossed over to our side of the road, raising the fear of their coming at us from that direction. There were a lot of shouts for our machine-gunner to set up in those woods on our right, which was soon done, with their firing over our heads into the woods beyond.

We all got busy digging in where we had jumped, which wasn't easy as we couldn't stand up to do any shoveling. The first night was a horror, with wounded strung out along the gully, able to receive assistance from the company aid-men (real heroes), but unable to be brought back to an aid station. The man on my right had a bad leg wound. He and others were dragged along in the gully into the woods on our left and then brought up over the wooded hill to be evacuated. While down along this roadside gully for a few days, this same route was used to relieve us a few at a time to get some sleep and heat up some rations in a farmhouse nearby, then returning to let others do the same.

As neither side was presenting much of a target at this time, the firing consisted of our machine-gunner and theirs firing sporadically across over our heads. At one point they got firing back and forth with the familiar ``Shave and a haircut'' rhythmic firing coming from our side, to be answered by the expected ``Two bits!'' from the German gunners. Even in war, there can be play sometimes.

Eventually, we were all directed to leave via the established gully path, to assemble and be moved forward at a different point, close to a nearby crossroads. I'll discuss the events that occurred there at another point.
\end{document}
