\documentclass[../m3y]{subfiles}
\externaldocument{../m3y}
% \graphicspath{{\subfix{../images/}}}
\begin{document}
There was marked shipboard excitement upon our viewing the English shore, going up the Channel to Southampton. We lined up with our full field packs, debarking and assembling on the dock, looking forward to, presumably, a trip through the English countryside to some inland American post, ``Forward march!'' then ``Left step, march,'' and we were oil a connecting pier, part of a long line going up the gangplank on a rusty tub that had probably been around since WW\thinspace I. It turned out to be an old English mailboat converted as a troop carrier. In the English navy style, we were assigned to hammocks suspended between posts, had a terrible meal of watery stew (also in the British military style), then sailed across the Channel under cover of darkness. Daylight found us at Omaha Beach on the Normandy coast, the scene of so much slaughter five months before. So much for merry old England; bonjour Ami!

Omaha Beach still showed a lot of the havoc wrought upon it four months earlier. A vast number of wrecked hulls had been chained together, extending as a long curve into the sea, this to provide a breakwater of sorts. We climbed down netting to the LST's gathered below, gazing up at the steep cliffs adorned with pillboxes on the brim, debarking in the surf. No wonder there was such a slaughter on D-Day. It must have been a scene straight out of Hell, and I take my hat off to the poor GI's that found themselves in such a situation\ldots\ absolutely true heroes, every one.

The first sight encountered upon scaling the steep climb up the cliff was a large American cemetery with line after line of white crosses. The majority of those buried there were probably returned later to the states, though many families chose to leave their loved ones where they fell. We bivouacked (made temporary camp) nearby in a mud field that could barely hold the pegs of our pup tents. The next day we were marched to a railhead where we boarded a long line of 40\&8's, the troop transport made famous in WW1. Forty-and-eights's are small boxcars (European rails are of a smaller gauge) designed to hold either forty men or eight horses. Simply bare boxes with only a couple of small openings once the door was slid closed, no toilet facilities (God help you if you had diarrhea), and nowhere near enough space for all forty men to stretch out on the bare floor at one time. This became our home for the next seven days, spending hours on sidings while other more urgent war material rolled by on the badly damaged railroads. Chow lines were set up only once in a while at these stops, as we were now getting our first extended experience with Army field rations\ldots\ K-rations were waxed boxes containing some canned meat, instant coffee, biscuits, chocolate, and a few cigarettes. C-rations were not much different, other than they were contained in cans rather than boxes. We were all given clever little can openers no larger than one's thumb.

Two events related to eating come to mind. At one stop they set up a chow line, the menu consisting of marmalade on bread and grapefruit juice, nothing hot. Saw no Red Cross personnel with their doughnuts and cokes.

At another stop, a Frenchman came by selling a loaf of fresh bread for cigarettes. The bidding soon went from packages to having men pool their rationed cartons. That  Frenchmen walked away with five cartons of cigarettes, the equivalent of \$120 on the black market, while the single crusted loaf was carefully divided between the few men who anteed up their cigarettes. No disputes but plenty of envy from the rest of us.

After a week of traveling, our Eurail passes must have run out, arriving near a caserne (former French military post), remaining there for a few days, then boarding another assemblage of 40\&8's for another five days of scenic touring before arriving at a second caserne, which the Army called Replacement Depots, the GI's ``Reppo Deppos.'' One event remains to this day: several of us were assigned to clean up after some departing groups and, to our delight, we found a full GI can of Spam, something I had eaten many times before in training. Three of us divided and consumed the can on the spot. The reader shall be spared any details on the consequence of this gluttonous endeavor; suffice to say that I have never been able to eat the stuff since.

It is of interest to note that at some point we turned in most of the equipment each of us had carried on board ship, this being the Army's clever way of using the available manpower to transport military supplies. We were issued other equipment upon joining our new units.
\end{document}
