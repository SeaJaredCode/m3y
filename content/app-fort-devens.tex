\documentclass[../m3y]{subfiles}
\externaldocument{../m3y}
\begin{document}
\vspace*{-30pt}
\begin{center}\large Ipswich, March 31, 1996\end{center}
Today's Sunday Globe contained a story as to how the military flag at Ft.\@ Devens will be lowered at 4:30 this afternoon and, thereby, the post will cease to be a military base, one which has seen all the heartbreak and happiness generated by two world wars and the ventures that followed.

I reported to Ft.\@ Devens in October 1943, an eighteen-year-old draftee just out of high school and with no more skills and developed talents than any of the other thousands who reported weekly, most to be assigned as foot soldiers in the infantry, as was my destiny.

A few things stand out in my memory of that day. We had reported previously for the routine physical and psychological tests administered in determining one's readiness for the demands of military service. Having passed that phase (one would have to be in deplorable shape to fail), I and about 50 others were instructed to report to the Court House at some very early hour, to be transported by bus to Ft.\@ Devens and, in most cases, further assignment to one or another of the military bases sprinkled about the country, most in the south where the weather permitted year round training.

Jack, my closest friend and classmate throughout all 13 years of public schooling, lived just a few doors down the street from our house. Not wanting to increase the pain of departure, my parents had accepted my decision to join with Jack in going downtown by bus, thus scheduling our family goodbyes to take place in the kitchen at home. These went as well as could be expected with all showing stiff upper lips and without any tears, though we were choked up a bit. My mother kissed me, one of the few times that she had done so, though I do believe she cared mightily.

\begin{figure}[t]
\centering
\includegraphics[width=0.8\textwidth]{fort-devens}\\
\medskip
{\newtimes\textsc{Fort Devens}}
\end{figure}

I walked down the street with my new toilet article bag (never had need for one before then) and rang Jack's bell as planned. To my surprise he did not respond in any way and after a second ring I gave up and went to the bus stop in front of his house, as the bus was due momentarily. His later explanation that he overslept may have been a cover-up for avoiding our own difficult farewell.

Just as the bus appeared at the top of the hill out shot my father in our '37 Chevrolet truck, the family limousine, pulling up neatly in front of the bus and beckoning me to climb in. I guess we were both afraid to talk very much lest our emotions conquer the thin fabric beneath which they were lying, and thus we rode downtown, making mundane observations about the weather and speculating about my chances of retuning from Ft.\@ Devens for a brief home visit before being ``shipped out'' the universal phrase of the day.

Bill Jubinville, a close friend of my father, was at the Court House to see me off, as he was for each of my brothers upon their departure to Ft.\@ Devens. His presence helped ease the situation and, following roll call we were soon boarding the bus. A firm fatherly handshake (hugging was not in his makeup) served as our goodbye, and I, of course, was the epitome of the stoic soldier off to do his duty (all 125 pounds of me).

I can remember when my older brother departed for duty in India two years before my being drafted \ldots\ how stunned I was to hear my father weeping in the bathroom later that day, though there was no display of hurt in our presence.

I just know that a careful listener could have heard a few muffled sobs coming from the bathroom later that October day \ldots\ the one when he drove me down to the Court House.

\end{document}
