\documentclass[../m3y]{subfiles}
\externaldocument{../m3y}
% \graphicspath{{\subfix{../images/}}}
\begin{document}
The word we heard was that the Army was getting a lot of bad publicity about the number of youths being killed in combat, with much of the criticism coming from Walter Winchell, a well-known journalist and radio personality, and had instituted a temporary policy of delaying combat assignments until a soldier passed his nineteenth birthday. There must have been some truth to this as all of us who were delayed from our overseas assignment were still approaching that date. And how else could the Army have us marking time other than to just start us on another basic training schedule?

Aside from my later combat experience, Camp Van Dorn was the worst assignment I had in the Army, being the pits in many ways. The barracks were one-story buildings clad in black roofing tar paper, which made a lot of sense in the searing Mississippi sun. The moisture-laden heat was tremendous, and we slept under musty mosquito bars to keep us from getting eaten alive by the mosquitoes. The training was heavy, and soon we were doing full 25-mile hikes with the regular 60 pounds plus of rifle and equipment. Thankfully, these hikes were started in the middle of the night so that we were finished well before noon, but that's the only concession that I can recall. One sight that remains vivid in my mind was that, while on a night bike, we saw a large burning cross on a distant hill, not uncommon where the Ku Klux Klan was still a local force.

The camp seemed to have an air of vague discomfort and corruption about it, which was confirmed when we discovered that our company orderly room was involved in some kind of bootleg liquor operation. It was off-limits to us, but there was a steady stream of and officers from other units making purchases. Our non-coms were mainly ``good old boy'' rednecks from the South, with an obvious antipathy to those who didn't fit that picture, often going out of their way to add more ``chicken'' to barracks life, tales not worth repeating.

Two things helped a lot during that period. We were allowed a weekend pass after completing our first six weeks, and Paul and I spent it in Natchez. We went from touring a couple of magnificent southern mansions along the Mississippi to, that evening, getting more than a little tipsy on beer, releasing the pent-up steam incurred in our unnecessary second basic training. Not an experienced drinker, I can still remember the hotel room swirling around after I went to bed.

The second, and more important savior of my sanity, was being allowed to play hymns on the Hammond organ in the chapel in our company area. This was my first experience on the organ, going over there almost every evening whenever the chapel wasn't being used, even getting good enough to use the bass pedals on the slow hymns.

Sometimes some soldiers would enter the chapel below for a few quiet minutes listening to my playing of the various catholic and protestant hymns. One evening a soldier climbed the stairs to the choir loft, sitting quietly behind me. Soon I heard him crying softly and, upon asking if he was all right, I learned his sad tale. It seems that, though married, he ``shacked'' up with a girl at a previous post and, since then, had been writing steadily to both this girl and his wife. When his locker began to overflow with their responses he gathered up all of their letters and mailed them back for their safe-keeping. Unfortunately, however, he mailed the package intended for the girlfriend back to his wife and sent her letters back to the girlfriend, and he had just learned that day that his wife was suing for a divorce. He was beyond any hope of assistance from me.

I am reminded of another domestic situation that occurred back in our basic training days at Camp Croft. Crawford, one of our platoon members who was quite a heavy drinker was excited because his wife was arriving by train on Saturday night, enabling them to have one night together before she returned to work. Paul Love and I wished him well as we left for a movie at the post theater on Saturday night. Upon returning after the movie we found, to our surprise, Crawford sitting on his footlocker, shedding heavy tears of regret. It seems that he had several beers while waiting for his wife's arrival and, upon her seeing him more than a little plastered, she immediately stepped back aboard the train without so much as a kiss. So much for his weekend of bliss.

This Mississippi assignment finally concluded and I was shipped up to Camp Kilmer, New Jersey, an assembly area for overseas shipment. While there we (Paul and I were still having parallel assignments) got a pass to spend a Sunday in New York City. A highlight was a free tour of Radio City, including witnessing a live broadcast of Phil Spitalmy and his all-girl orchestra, after which I was able to call home. I carefully followed instructions by not telling my parents that I was about to go overseas, though it was self-evident when I told them about the radio show I had just seen. A few days later we were brought over under cover of darkness to board ship someplace on Staten Island, soon sailing out to join a large convoy being assembled on the high seas.

I was not a happy trooper, being disappointed that we were not allowed, for security reasons, to go on deck for a final glimpse of the Statue of Liberty and the skyline, but more distraught in that Paul was not on board, having been delayed for a further eye test. I was never to see him again as, after serving in combat in Italy, he returned to civilian life as a student, from which he disappeared under mysterious circumstances unknown to his family or classmates, never to be seen again. In my follow-up in later years, including visiting his mother in Illinois, it appeared plain that he was somehow a victim of foul play.

Adding further insult to injury, my achieved MOS rating as a communications specialist was now superseded by that, once again, of an infantry rifleman, as that was my latest training designation. Oh, well, I was learning that the Army way was the only way and a hard one at that.
\end{document}