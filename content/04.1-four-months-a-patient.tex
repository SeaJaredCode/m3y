\documentclass[../m3y]{subfiles}
\externaldocument{../m3y}
\begin{document}
I had a pleasant surprise upon my first night in the large American General Hospital in Paris. In its thorough approach to accounting for all casualties, the Army had clerks checking our dog tags at each new medical facility, duly recording the details. A clerk performing that duty was slowly proceeding from one cot to the next and, as he got closer, I recognized him to be David Lane, a neighbor in our Highlands section of Lowell, and one of my high school classmates. When he asked me my name, I responded with ``Bernard Huntley of Lowell, Mass'' which produced a puzzled look on his face as he tried to connect me with his Lowell days. At this point I had not had a haircut for over two months and, no doubt had lost some weight off my normal skinny frame \ldots\ a bit of a sight. We had a great exchange about Lowell and discussed our individual military experiences. Later he returned with a couple of candy bars and some paperback books, which were called pocketbooks in those days. My parents were able to get a direct report on my condition through his subsequent letter writing to his father, a fireman who was a regular customer in our hardware store, which must have relieved them a bit.

On that subject, my parents had received a telegram from the Dept.\@ of Army saying that I had been slightly wounded in action, with details to follow, which is now in my possession. I got a note off to them as soon as I could, suggesting they hold writing to me until I got my new hospital address.

\begin{figure}
\centering
\includegraphics[width=0.8\textwidth]{telegram}\\
\medskip
{\newtimes\textsc{Telegram to my Mother, January 1945}}
\end{figure}

From Paris we went by hospital train, emblazoned with large red crosses on the roof to protect it from enemy strafing, to a French port, probably Cherbourg or Le Havre on the English Channel. The railroad cars on a hospital train consist of a double line of racks the length of the car, upon which our cots could be stacked, four of five high as I recall. This crossing of the Channel was very different from the one of the previous October in that the ship was all white, like a cruise ship, prominently illuminated at night to display the large red crosses painted on the sides. Bright, cheery, ultra-clean, warm, and with hot meals \ldots\ plus pretty American nurses. A delight, lifting our spirits considerably.

Our next stop was the \nth{7} General Hospital, outside of St.\@ Albans, about twenty miles from London, going again by hospital train from the port. The \nth{7} General was comprised of a number of Quonset huts, ward-sized rounded metal buildings, shaped as if made by cutting a large metal pipe in half, each elongated section resting flat on the ground. These were heated by two G.I.\@ issue pot-bellied stoves, which could really sometimes glow red on the sides. Each building was a typical ward, with hospital beds lining each side and a toilet and two small rooms at one end. All huts were joined by open wooden roofed walkways, spreading over what must have been open countryside before the war. The ward was what you might expect, with hospital beds lined up on each side.

What distinguished our ward, and several others, was that the beds were made as if they were going to be slept in from either end, this to permit our ugly damaged feet to extend in the air from the bottom. Any contact with bed clothing was quite painful. We must have been a site, displaying our feet as if they were trophies… I guess they were combat trophies in a way.

I soon discovered that most of the hospital staff was made up of doctors, nurses, and aides from the Boston City Hospital, having enlisted as a group to assist in the war effort. Even the priest had been their hospital chaplain. Even better, the prettiest nurse of the lot was one of the two assigned to our ward, the other being a rather plain-Jane of Polish extraction, who really bossed us around. Initially, we were all gaga over the cute one, but we soon learned that her heart wasn't devoted to our care, where the other nurse was a gem in that respect. She was the one who started right in washing up her grimy bedridden patients, hair shampoos and all. Speaking of hair. I had so much of it at this point that the guy in the next bed, a hillbilly from the South, called me ``Railhead'' a nickname that stuck with me for the next several months.

\begin{figure}[h]
\centering
\includegraphics[width=0.8\textwidth]{patients}\\
\medskip
{\newtimes\textsc{Typical evacuation hospital}}
\end{figure}

It seemed that not much could be done to accelerate the healing process of trench foot, the main concern being gangrene. Several men lost feet, usually only one, and many up and down the ward lost toes. The toes on my right foot were of particular concern to the doctor, with the large toe as black as tar. At one point they had improvised an iron lung, a device which surrounded one's body, with just the head and feet projecting, designed to produce powerful internal pressure which compelled the patient to breathe deeply as the machine went through its machinations. The idea, apparently, was to stimulate the blood flow to your legs and feet. Not painful, and probably not very helpful, but I'd give them an ``A'' for effort.

I was to remain at the \nth{7} General Hospital until the following May, bedridden at first, gradually healing as the months progressed. Many in the ward, having lost limbs, were shipped home for discharge, the rest of us passing much of the time by reading, playing cards, Monopoly, and checkers, having bull-sessions, etc.\@ eating our meals in bed. It is of interest to note that, while we were keenly following the progress of the war, there never were any discussions of our combat experiences, but much talk about home and families and our plans for the future. BBC and AFN programs were piped into the wards, which is how I became familiar with the wonderful Welsh male choirs that were broadcast on Sunday mornings.

My ugly right foot gradually grew new skin under the black parts and, one fine day, the whole blackness sloughed off my toes completely, revealing bright new pink toes underneath. Not long thereafter I was able to become ambulatory --- a blessing to do away with the bedpans --- and soon I was recovered enough to go to the Mess Hall for my meals. By this time I was receiving mail and, sometimes, packages from home. Soon I was doing well enough to get passes for a day in London, later getting a full week's leave in London before being shipped back to France.

The poets have written rapturously on the loveliness of springtime in England, and the spring of 1945 was especially beautiful, possibly to make up for the horribly cold preceding winter. Our hospital was a couple of miles outside of St.\@ Albans proper, with our route to the train station taking us through fields and small woods. The wildflowers were profuse, the bees having a holiday, the sun bringing rich fragrance from all that was growing around us \ldots\ really special. I remember that spring very well, indeed, particularly the walk to and from St.\@ Albans. As the feller' says, ``Ain't it good to be alive?''

Another benefit to finally getting on my feet was that I was able to take advantage of some of the field trips that were arranged by the Red Cross personnel. These included a visit to England's movie-making center, where we stood behind the cameras, watching Cecilia Parker and some other star filming a scene, which I got to see in its entirety a few years later. Another day we toured Cambridge University where we were given tea by one of the dons in the very quarters occupied by Erasmus, centuries before. A Sunday afternoon's trip was to a large theater in London where we were given a marvelous show by the stars of the British musical stage.

Our local English pub experience centered around a small pub just outside our hospital area. This was a most friendly place, with the locals often insisting on buying ``a pint for the Yanks.'' The resident pianist was missing some fingers, but that didn't deter the crowd from joining in lustily on ``I've Got a Sixpence'' and all ten verses of the naughty ``Roll Me Over in the Clover.'' These visits gave us a good impression of average, working class English folks, freely extending their hospitality and friendship to us.

Another example of this (with an ulterior motive, however) was when another GI and I were given a written invitation to dinner in an English home, this being delivered by the two young lads who visited our wards regularly, sometimes bringing flowers, singing for us, and always providing a bright note of cheer to the patients. The invitation came from a very modest family, mother and two children, with Daddy away in the war, living in one of the many narrow row houses that are typical of English cities, with dinner served on their very best crockery, the entree concealed under a covered platter. After a non-alcoholic toast, the cover was lifted with a flourish, revealing a fine looking square of Spam (which I could no longer stomach!). They must have used a lot of their meat ration for the dinner and we responded as if it were the rarest of roasts.

The ulterior motive behind all this became apparent when we were introduced to the eighteen-year-old daughter \ldots\ obese, dyed blonde hair, and giggling constantly. Mother encouraged us to accompany her to see their small vegetable garden. ``You young folks go enjoy yourself.'' There were hints about how much fun the picture shows are, etc.\@, but neither of us rose to the bait. Still, it was a nice thing they did for us.

One of the things the hospital did was to provide transportation to a large roller skating rink, encouraging our participation as they thought it would help blood circulation in our limbs. I was amazed to see that live music was provided by four real old musicians, ensconced on a small balcony high above the floor. There were a lot of unattached English girls skating and, once I got the hang of it, I started giving a lot of attention to a very pretty (and slim!) English girl who was extremely shy around GI's. When I skated with her a week later I got up the courage to ask her out to see a movie. She said her mother would not approve but, after a bit more skating, she agreed to meet me at the movie theater the following weekend \ldots\ minus mother's blessings. It was a nice time but she was so nervous that she wouldn't even hold my hand in the dark. I wasn't even allowed to walk her home. Oh, well, she blew her chance to become a GI war bride!

At one point toward the end of my hospital stay, I drew KP duty, reporting to the mess sergeant very early in the morning. Somehow we discovered that we were both from Lowell (maybe my New England accent was noticed), which turned out to be a most fortunate thing as, instead of having to put in an unpleasant day of kitchen duty, he brought me across to his quarters, where I spent a most pleasant day reading his magazines, curled up cozily by the warm fire. After the war this fellow, who was of Greek extraction, returned to his post as some sort of junior executive for Necco Candies in Cambridge, and I ran into him almost every day when I was commuting by train to Boston University.

As the month of May approached, it was determined that I needed no further medical treatment, though I was placed in limited service, which meant that I would not be returned to front-line combat service. As was the standard practice, I was given a leave, which I spent taking in the many sights of London. The Tower (the royal jewels were not on display due to the war), St.\@ Paul's, and Buckingham Palace. German V-Bombs (rocket-propelled) were still arriving in London but, as interception wasn't possible and their targets were random (one landed near our hospital), the British didn't sound the sirens nor bother with using the air raid shelters. These were pretty powerful weapons, however, being easily heard upon landing at a distance. One such came in while I was listening to a tour talk by one of the Beefeaters at the Tower, causing him to hesitate a moment as if to reassure himself that it wasn't close by, then proceeding as if nothing had happened. London, of course, still displayed terrible scars from the Blitz of months before, with hundreds of homeless people spending their nights on pallets deep down on the subway platforms. Passengers walked by these sleeping hordes just as if they were the most normal of sights. The British are a gritty bunch, all right.

Along with other GI's, I lodged at the Red Cross dormitory close by Speaker's Corner at Columbia Arch. These soapbox orators really drew the crowds on Sunday mornings, and I got a kick out of them. Most speeches were on politics and the war. One man, however, led a small crowd in singing various songs of the day, ``Bluebirds Over the White Cliffs of Dover,'' ``I'll Be Seeing You,'' ``We'll Meet,'' etc\@. I was told that he was a successful businessman who did that every Sunday morning as his contribution to the war effort.

Entertainment recalled includes the stage play \emph{Arsenic and Old Lace}, in which the British actors interpretation of a wise guy New York newspaper reporter was really corny, and seeing Judy Garland in \emph{Meet Me In St.\@ Louis}, in a handsome mammoth West End movie palace. This one still had its powerful pipe organ, with the spotlighted organist and his all-white console rising dramatically out of the orchestra floor to provide a few numbers before the show started, a practice once common to American movie houses.

The center of American GI's in London was at the Rainbow Club in Piccadilly Circus, which was also the hub for buses to the various Red Cross dormitory facilities spread around London. One of the fellows I met at my dormitory invited me on a double date that he had set up with two British nurses. Movies, a bite to eat (I became very fond of English pork pies, often eaten cold), and walking them back to their dormitory. A bit of friendly necking, then returning for transportation to Piccadilly Circus late in the evening. A most vivid memory, however, was that when we got to Piccadilly Circus the newsboys were shouting ``Extra \ldots\ extra \ldots\ President Roosevelt dead!'' What a shock! Not only had he been our great wartime leader, he had been my president since the time when I was seven years of age. The effect was every bit as stunning as the death of President Kennedy, years later, and I imagine that any person of my age can recall exactly what they were doing when they heard the news.

Following my fine week in London, I returned to the hospital from where, very shortly, I departed to report to an Army base for processing and, within days, I was once again crossing the Channel, this time to a replacement depot in Le Havre, awaiting the Army's desire for my next assignment.
\end{document}
