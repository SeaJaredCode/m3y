% !TEX root = ../m3y.tex
\chapter*{Forward}
\externaldocument{../m3y}
\thispagestyle{plain}
\addcontentsline{toc}{chapter}{Forward}
\label{forward}
Of my father's many facets --- teacher, intellectual, musician, storyteller, dreamer, traveler, lover of fun and beauty, husband, father, grandfather and more --- war veteran, while not hidden, did not seem as prominent until his last years. Now it is clear that his time as a soldier in many ways determined the arc of his life and legacy. The story of his coming of age as a soldier in WW\,II is both universal and unique.

During the war, exposure to educated comrades opened his eyes to his own intellectual potential and he was able to act on that newfound awareness courtesy of the GI Bill, ultimately and improbably earning a doctorate from Harvard. He emerged from the war convinced that in children lay the hope for the future and he was drawn to a career as an elementary school teacher, a choice that was as idealistic as it was unusual. Exposure to foreign people and countries during the war instilled a lifelong wanderlust and also led to spending a decade in Germany living as a civilian on army bases. My father's war was a matter-of-fact part of our lives as indeed the war was the backdrop of all life in Europe in those days. We traveled to places he had been during and after the war, visited Little Joe the war orphan he adopted for a time, laughed about his wartime English ``girlfriend,'' enjoyed his annual tale of the Christmas he spent in a foxhole and occasionally happened upon the handgun he took from the German soldier. He rarely mentioned the dark side of his experiences though we somehow knew he didn't place much stock in war. He never thought his part in the war was in any way heroic, though he did use the fact of his war service strategically to establish common ground with military personnel.

When we returned to the U.S., the war receded almost completely though he continued to make treks to his wartime haunts and once made a special trip with Allan similar to the one the whole family took.

When Grandpop (as he was known by the next generation) retired, the war thread was picked up again and it was veterans with whom he wanted to spend his time and energy. He often said that playing the piano at SoHo (Chelsea Soldiers' Home) was the most rewarding thing he ever did.

And when he died, his war memoir was what defined our father and grandfather in his own words. Almost all of the many obituaries highlighted either his WWII experience and our family trip or his playing at SoHo or both.

\begin{flushright}
Betsy Huntley\\
February 19, 2011
\end{flushright}
\clearpage
