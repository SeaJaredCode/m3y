\documentclass[../m3y]{subfiles}
\externaldocument{../m3y}
\begin{document}
After a few days at the second Reppo Deppo, I was pleasantly surprised to be assigned as a radioman (someone must have seen my training record) in an artillery unit (105 howitzers, as I recall) which was in the thick of fighting in the Saar area. The next days were spent running communication wire before various points, sleeping on bins of sugar beets stored in the cellars of the abandoned French homes, getting used to the terrific noise of the firing weapons, and making the necessary adjustments to life in a combat artillery unit. While no bed of roses, it was good to have a line of infantrymen between us and the enemy. I was not unhappy, either, to be carrying a carbine, the personal weapon of artillery personnel, instead of the heavier M-1 of the infantry (a fine weapon, incidentally).

Two events stand out in my memory of these weeks. At one point we were at some forsaken Saarlach village with a single street running up the hillside, houses one against the other, each with an enclosed barn area under the same roof and, inevitably, with a huge mound of animal and human waste mixed with straw in front of every doorway, this being saved for spreading on the fields during the growing season. A couple of us went into one of those barns (I no longer recall the reason for this), to discover that it was overflowing with the dead bodies of American soldiers. Every inch of hay, the barn floor, even the hay wagon, was covered with bodies. Obviously, it was a collection point awaiting the further action of a graves registration unit that would see to their proper burial. I guess the freezing cold weather kept the smell down; in any event, we didn't waste any time in getting out of there. Terribly sobering \ldots\ guys my own age \ldots\ reminded me of some of the Matthew Brady photographs I had seen of the Civil War, and others of WW\thinspace I trench warfare. The thought that this was just one small segment of the price that was being paid daily in Europe and the Pacific has influenced my attitude on war ever since \ldots\ the foolish tragedy of the whole business, the human waste of it all.

The other event of some interest occurred when I was accompanying our platoon sergeant in a jeep, delivering some papers to our headquarters. It was a very dark night and we were going along without any lights. At one point we came close to hitting some kind of barrier, so close that the tip of my carbine, which was on my lap and projecting outside the vehicle, got whacked pretty good. Later, I discovered upon cleaning the rifle that the barrel was slightly bent, which would have presented an interesting development had I occasion to fire it. I was issued a new weapon the next day.

\medskip
\begin{figure}[h]
\centering
\includegraphics[width=0.8\textwidth]{nyt-12-2-44}\\
\medskip
{\newtimes\textsc{We Made the News!}}
\end{figure}

\end{document}
