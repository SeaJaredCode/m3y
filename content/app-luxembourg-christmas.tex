\documentclass[../m3y]{subfiles}
\externaldocument{../m3y}
\begin{document}
\vspace*{-30pt}
\begin{center}\large Battle of the Bulge, 1944\end{center}

\begin{figure}[t]
\centering
\includegraphics[width=0.8\textwidth]{koetschette}\\
\medskip
{\newtimes\textsc{The Inn in Koetschette (Now a Cafe), ca. 2020}}
\end{figure}

Christmas Eve was spent in the small Luxembourg village of \\Koetschette\footnote{Known to me at the time as Kochschette.}, which our company had secured with few casualties about mid-afternoon, and we soon settled in with considerable excitement for we would get to sleep inside instead of having to hack out foxholes in the heavily frozen earth. Our squad took over a small inn, the most prominent building in sight, and we were soon checking out what wine and other beverages had been left by the Germans who occupied the place until that day. I even got to bang out a few tunes on an old piano.

This welcome relief, the first in eight days, was heightened by the news that our mess sergeant planned to bring up his field equipment in order to cook a full feast on Christmas day \ldots\ roast turkey, mashed potatoes, apple pie and all the other traditional fixings. Any ordinary hot meal would have been a real treat but a full turkey dinner was almost beyond our comprehension.

It so happened that one of my two-hour stints at guard duty found me at the front door of the inn at midnight, hours spent without any of the usual incoming artillery or mortar fire as, apparently, each side decided to honor the solemnity and peace of Christmas Eve. I had plenty of time to think long thoughts about the folks at home and, of course, wondering about what might be coming my way in the days ahead. Some of the time was passed by saying the rosary, thus I acknowledged my heritage and, possibly, enhanced my heavenly insurance a bit.

True to their promise, the cooks arrived early the next morning, setting up their field stoves in front of the inn, with our anticipation increasing in proportion to the wonderful fragrances penetrating the frigid air. All of our excitement came to an abrupt halt, however, upon learning that there would be no noon meal as we had been ordered to move out and take the neighboring town. Though the mess sergeant assured us that we would still get our Christmas dinner regardless of how long it would take to secure that objective, our disgruntlement was palpable.

Maybe that was just the incentive we needed, however, as our task was accomplished by late afternoon and, as promised, the portable ovens were soon reheating the dinner in the middle of the town square. Once again we were fortunate to be able to stay inside overnight, this time in a farmhouse guarded by a rather skinny German Shepherd dog.

As the house was located somewhat out of town, our trek to the Christmas goodies required a fast run down an exposed stretch of road. Not wanting to miss anything, my mess kit was soon overloaded with so much turkey, mashed potatoes, dressing, and gravy that I decided I could do better if I made two trips. Thus, one half of my kit was left at the farmhouse while I went down for a crack at more of everything, plus dessert and coffee.

Upon returning to the farm house the second time without incident, I discovered my aluminum mess pan as clean as if I had polished it while, close by, one very happy German Shepherd was licking his chops. My buddies told me that he was into it before they could stop him so they let him finish it off.

This all ended well, however, as I was able to replenish my dinner without any problem and I did it full justice, indeed. Later, while dozing off most contentedly, I got to thinking about how I had spent Christmas Day when it occurred to me that I had, after all, been able to give at least one present that Christmas and, while I did not plan it what way, it could not have been received with more relish or appreciation.

Though this completes my Luxembourg Christmas story, you may be interested to know that I have twice revisited the scenes of my days in combat and even got to patronize that Christmas Eve Inn. All is most peaceful there now, roofs have been rebuilt, shrapnel pockmarks have been stuccoed over, and a rural calm prevails. I am pleased to report, also, that the local citizenry expressed genuine appreciation for what some American lads did for them many Christmases ago.

Merry Christmas and love to all of our dear friends, and let us hope this tired old world can apply a few hard-earned lessons in the brand new century ahead.
\end{document}
