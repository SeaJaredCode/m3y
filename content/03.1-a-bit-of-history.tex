\documentclass[../m3y]{subfiles}
\externaldocument{../m3y}
% \graphicspath{{\subfix{../images/}}}
\begin{document}
Historians describe the Battle of the Bulge as the most significant battle of WW\,II in Europe, essentially turning the tide on the war on the western front. It was also the most costly, with the Allies losing 77,000 men (8,600 killed, 47,000 wounded, 21,000 missing), while the Germans suffered the loss of 82,000 men (13,000 killed, 39,000 wounded, 31,000 missing). Lasting almost exactly 4 weeks, with 159,000 casualties collectively, or 40,000 each week, provides one dimension of the intensity of that particular battle. Staggering when one considers its effect on those who survived the loss of their buddies, to say nothing of the ever-widening circle of grievous loss suffered by families and friends in the states.

To back up a bit for the benefit of those for whom WW\,II is not a memory but, truly, ancient history, it is necessary to recall that D-Day, June \nth{6}, occurred six months before the Battle of the Bulge. This beginning was brought home for the youngest among us through the recent motion picture, Saving Private Ryan, which made the invasion and the gore of war particularly vivid.

Going on from there, it took the Allies about a month and a half to break through the German defense of Normandy, this at St.\@ Lo. Once freed up, the Allies moved rapidly through France, particularly the \nth{3} Army under General Patton, chasing the retreating Germans back to, essentially, defending their own borders at what has been termed the ``Western Wall'' where their resistance stiffened considerably, bolstered by man-made and natural obstacles, but mainly through the understandable German desire to protect their homeland at all costs. Contributing to what might be called a stalemate, though a lot of fighting was still taking place, was the need for the Allied Forces to hold up until their supply lines could support a major thrust into Germany itself. We had essentially outrun our supply capability due to the welcomed, though unexpected, rapid dash across France. There was a severe need of replenishing our troop losses, as well as assembling the tremendous amount of gasoline, ammunition, food, and other supplies critical to the next big push through to and over the Rhine, and into the heart of Germany.

In their optimism over the routing of the German forces from France, our leaders felt that Germany was in no position to launch a major offensive, busily focusing, rather, on preparing for a massive defense of their homeland, thus we concentrated the strength of our armies on those areas providing the stoutest opposition. The Ardennes Forest, spreading over Luxembourg and part of Belgium, with its sparse roadwork of narrow roads, was seen as the area of least risk, thus our troops were spread particularly thin along the line facing the German border, with a prevailing air of R and R before the coming big push scheduled for mid-December, that to be headed by George Patton's Third Army. He was no doubt chafing at the bit over the coming opportunity to once again display his considerable war skills.

It was during this period that I tasted combat in the artillery, then was transferred to the infantry, becoming one of Patton's replacements in Metz where, as one historian put it, the Third Army really got its nose bloodied.

To talk about Hitler for a moment\ldots\ he has been amply described as the fanatical race-mongering dictator that he was\ldots\ so much so that he is almost an unreal, or mythical, figure in the public mind. One could easily overlook his brilliance and the fire in his bosom that he still had for his beloved fatherland. After a serious attempt on his life by a group of junior officers in August, he spent many weeks in bed recovering from the wounds received from the bomb that had been placed almost at his feet. Historians record a marked growing paranoia in his personality at this time, causing him to assume even greater personal control of every action of the military, and a growing reliance upon his intuitive judgment, forsaking any words of caution by his senior generals.

This is when he conceived what was to be his final masterstroke, when he would split the Allies asunder and regain the war initiative, as well as providing time for the further development of his significant new weapons --- jet-propelled aircraft, rocket weapons, etc.\@ --- by which he expected to annihilate the enemy forces, bringing our politicians to their knees.

This masterstroke was, of course, what came to be called the Battle of the Bulge. Hitler's strategy was to spend the fall months in assembling a massive force of men and weapons, under the strictest secrecy, attacking the Allies through the most unlikely terrain, the Ardennes Forest, at a time when the weather conditions would not permit significant action by our air forces, relying on surprise and superior mobile forces to penetrate the American and British forces right through Luxembourg and Belgium to the sea at the great port of Antwerp, vital to our supply requirements. This would completely divide the Allied forces between the north and south, providing the opportunity for vast military gains in the chaos he would generate. This would require the amassing of every bit of manpower, tanks, artillery, and other war requirements, accomplished in the knowledge that the very survival of Germany would depend on the outcome of this final offensive. This, undoubtedly, contributed to the determination and ferocity of the average German soldier\ldots\ the welfare of their country, their families, and their future was dependent on this one grand effort to turn the war around.

Thus, in the early hours of December 16th, the mighty German military machine was, once again, on the march\ldots\ and their grand plan succeeded as intended, catching us sound asleep, creating tremendous panic and chaos, and actually, within days, penetrating an unbelievable 70 miles within our lines, heading as rapidly as possible for the complete severance of our northern and southern armies through reaching the sea at Antwerp. If successful in completing his march to the sea, Hitler anticipated a mass rout of the British forces, compelling their return to England, a falling out of the uniform allied front and, also, he expected to have sufficient time to complete the development of new rocket fired weapons and jet aircraft, weapons which he promised would bring a rapid vanquishing of Germany's enemies.

Hopefully, this brief historical review will suffice to provide a general understanding of the dire situation created, one which presented the most serious challenge imaginable to our possibility for victory in Europe. Skipping over details that would be required were this to be an extensive historical account, it can be said, in summary, that once the initial shock was overcome, our military leaders responded brilliantly, the fighting men heroically, in facing and overcoming this newly inspired and ferocious enemy. General Eisenhower, supported by Generals Montgomery and Bradley, quickly set about redeploying troops to bear down from the north and up from the south in an attempt to slice through the vast finger of the German penetration. The first to swing his divisions from the western wall to the north was, as one might expect, General Patton, earning a special citation in all historical accounts for the skill and speed displayed in swinging his forces north into Luxembourg. Thus, shortly before Christmas 1944, I found myself embarking on my first direct front line experience, this in the rear of a little cemetery in Nothum,  not knowing what to expect as we started down that hill in the eerie and short-lived quiet of that particular morning.
\end{document}
