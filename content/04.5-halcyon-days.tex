\documentclass[../m3y]{subfiles}
\externaldocument{../m3y}
% \graphicspath{{\subfix{../images/}}}
\begin{document}
If my days in England and Calas Staging Area were great compared to life as a combat infantryman, my final half-year in the Army, spent in Aix-en-Provence was truly halcyon. Upon the closure of Marseille as a major port of embarkation due to the cessation of needing to send troops through the Suez Canal to the Far East, the extensive tent city at Calas, holding up to 70,000 men, was folded away and the massive plain allowed to return to its previous condition\ldots\ wind blown, desolate, and covered with weeds when I revisited it years later.

Once again my one-semester high school course in typing, plus my experience in the Troop Movement Office at Calas, continued to serve me well, resulting in my assignment as a clerk in the Officers' Records Section of a Graves Registration Unit headquartered in Aix. While administered from there, the unit's real work was done out in the field and at a major laboratory located further north. Its' task was to recover bodies, American, mainly, but also German, French, or any other soldier buried in rural graves all over southern France. Field units would, as a first step, put up posters in several neighboring towns, explaining their mission and giving notice that a team would arrive a few weeks later to interview all who had some knowledge of a buried soldier or airman. Following these interviews, teams of German POW's would disinter the body, bag it, and send it to a regional lab for identification using anything found such as dog tags, dental configurations, etc. Following substantial paperwork, the body would either be returned to the states for burial or to a U.S.\@ military cemetery in Europe, with other bodies transferred to other national jurisdictions.

Despite its grisly, but vital, mission, the headquarters of the unit could not have been in a more attractive and peaceful setting. Aix, being the seat of a university, is a beautiful old town, known in particular for its tree-lined Rue de Maribeau leading down to a large and elaborate fountain in the center of town, one which would rival some of the major fountains at Versailles. Rue de Maribeau is a wide boulevard, completely shaded by two rows of plane trees, and lined with solid French architecture, three stories high, with several excellent sidewalk cafes and restaurants within the fountain area. Its beautiful proportions and classical ambiance continue to attract tourists to this day.

After progressing sequentially from foxhole living, to hospital ward, to 16-man tents, our quarters in Aix appeared fantastic. The entire unit was located in a single seven-story apartment building close to the center of Aix, with our offices on the first two floors, a WAC unit (female Army personnel, mainly black) on the next, non-commissioned officers above them, followed by several floors of GI's (I was on the fourth floor) and, finally, with the top floor assigned to German POW's, who performed many routine duties within the building and in our company mess hall, a prefab erected behind the main building. Our Red Cross Club, our major social center, was also in a prefab to the rear of our headquarters. My room, which must have been a bedroom originally, was furnished with 3 GI cots and 3 footlockers\ldots\ pretty sparse, but high-class living me, Bill Martin, and Little Joe. Its chief attraction was a small balcony which provided a broad view of many miles, culminating in a great view of Mt. St.\@ Victoire in the distance, a feature of many paintings by Cezanne (who was born in Aix), Gauguin, and many other Impressionist artists.

Our officers' records section was staffed by four persons, the sergeant in charge, myself (now a corporal), Grace (a very nice black WAC), and Odette Rougier, a most capable local lady, in her 30's, with an excellent command of English. The work encompassed all the duties one might expect\ldots\ assignments, pay records, maintaining officers evaluations, etc\@. While there was enough work to keep one busy, it was a pretty easy assignment once the tasks had been mastered. The personnel were pleasant to be with, our Sergeant, being a combat veteran, had a light supervisory touch, and I could not have had a better assignment, nor a better location, than to be in Aix for my final months of Army life.

A few fond recollections:

\section{TBD}
If our work for the day was accomplished, Bill Martin, Jim McClinden, or any others who could get off work early, plus Little Joe, could knock off around four thirty for a glass of wine at one of the sidewalk cafes on the tree shaded Rue de Maribeau. One cafe had an excellent instrumental trio consisting of piano, guitar and string bass, which gave us a lot of pleasure. Sooner or later they would play ``Symphony, Symphony of Love'' which was a popular hit, and the pianist was always asked to render DeBussy's ``Clair de Lune'' which I shall always associate with that particular spring in Aix.

\section{TBD}
D.J.\@ (I think her last name was Anderson) was a truly pretty and vivacious activities director at our Red Cross Club. She was given a lot of attention by the officers of the unit but, interestingly enough, she fell in love with my roommate, Bill Martin. D.J.\@ was loveliness personified, and Bill fell but hard. The two Red Cross girls were housed, along with our officers, across the street from us in the Roy Rene Hotel, a premier facility favored later by Winston Churchill when vacationing and painting in the area. Again, one must cite the unusual kind of Army life we were living after the war, when discipline was relaxed as long as we behaved and fulfilled our responsibilities, and there was a pronounced, but unspoken, comradeship between those officers and enlisted men who were ex-combat infantry veterans. As Bill and D.J.'s romance grew Bill would sneak over to her room at the Roy Rene (ostensibly very much off-limits to a GI), often returning to our room in the middle of the night. Assuredly, sympathetic night duty personnel at the Roy Rene must have looked the other way. It pleases me to note that Bill became a doctor after the war and married our beautiful Red cross Director.

\section{TBD}
In keeping with the relaxed military atmosphere, GI's were permitted the use of jeeps and \( \frac{3}{4} \)-ton trucks for their personal side trips around the area. One memorable sojourn took us high into the French Alps in an open jeep, with Bill and D.J.\@ up front, while Jim McLinden and I sat in the rear, where we would have frozen if it weren't for our having brought blankets, knowing that we would be traveling into high alpine snowfields.

Another favorite spot, which we visited several times, was Le Baux, a remarkable deserted village dating back to medieval times, perched high above the plain of Avignon.  At that time there was only one occupant in the village; today it is a major tourist attraction, larded with gift shops, small hotels, and restaurants. Still a worthwhile attraction, however, to which we have returned several times in the intervening years.

\section{TBD}
Though we were now somewhat dispersed, being assigned to assorted units in the area, Tony Germano, Howard Samp, and a few other guys from our Calas days, still got together every once in a while. One time the Army distributed confiscated vodka, a bottle to each enlisted man, this calling for a major party held in our Aix bedroom. Howard brought along some stateside corn for popping, we had grapefruit juice to cut the vodka, and I gave my accordion a real workout. Bill Martin always gave a solo performance of ``A Wandering Minstrel, I'' from \emph{The Mikado}, in which he had performed in a college production. I don't know how it came about, but we wanted to get something out of the neighboring bedroom, which was locked, however. Nothing would do but to cross over from one balcony to the other, four stories up, a sizeable gap between the balconies, with several of us doing it just for the fun of it. Viewing the site of this caper several years later brought home the similarities between our actions and those of any young men, such as fraternity brothers, in another setting.

\section{TBD}
It was at this time that I made a side trip to visit Harry Wilson, a valued employee in Huntley Hardware, now a sergeant in a medical unit. Harry was always an amiable sort, and we had a good visit, reminiscing about Cupples Square days.

\section{TBD}
The reader may recall that I lost my devoted pet dog on Christmas Eve of 1945. The story didn't end there, however, as one day toward the end of my stay in Aix, a few of us, including Little Joe, went down to the village center to see the attractions of the carnival setup, the first since before the war. One of the rides was the typical prewar chain swings suspended from a rotating frame, resulting in the swings spreading wide as the speed increased. We got up in the seats and, just before it started rotating, I was thrilled to see my lost dog in the crowd. I whistled for him and he came running over as the swings started to move. He circled, barking and following beneath me until the speed got too fast, settling for sitting and watching me circling above. At one point he was joined by a rather large matronly woman, who also stood waiting for the ride to stop.

Well, he was all over me once the ride ceased\ldots\ a most happy reunion. With Little Joe as our somewhat limited interpreter, the lady explained that her husband had brought the dog home on Christmas Eve, having ``won'' him in some kind of a raffle at a bar. The dog was obviously well cared for, his coat was washed and powdered, and he had a kerchief around his neck. Knowing that I would be returning home shortly and could not take him with me, I was pleased to tell the woman that she could keep him, which provided her with considerable relief, as she could plainly see that the dog was mine. Thus, there was a happy ending to this canine tale.

My speculation was that the one disagreeable tentmate we had, a heavy drinker who delighted in teasing the dog meanly, a source of strong words between us, leading to the dog's growling every time he passed my cot, must have deliberately taken the dog out to the bar on Christmas Eve, either giving him away or selling him for drinks. I'll never know for sure, but it does explain his disappearance.

\section{TBD}
One night D.J.\@ had arranged for a dance with some of the female university students. This was a remarkably chaste affair, the Army transporting them and their two chaperones to the Red Cross Club, with our being lectured ahead of time on our obligations as representatives of the U.S.\@ Army, and no couples allowed to wander outside the club. Doughnuts and Cokes were served, records were played (Glen Miller and other Big Bands), dancing took place, and, sharply at ten, the lovely students and their chaperones were returned to the university, without the opportunity for any private liaisons to develop, which is understandable in view of the general reputation of most GI's. It was nice to have attractive female company, however brief.

\section{TBD}
It so happened that I was serving as Corporal of the Guard on a day when a GI was being court-martialed for leaving his post while on guard duty at the main entrance to our apartment building. Under certain conditions this could have been a very serious offense, even warranting death in wartime. In his case, he just left for a brief time to join his buddies for a beer someplace, sufficient violation to warrant a finding of guilty and a sentence of six months in an Army stockade, again an example of the Army going easy on combat veterans about to be discharged. My job was to guard the accused during the trial and then to see to his subsequent deliverance to the stockade, limiting his accompanying baggage to clothes and toilet articles only. He happened to have a relatively new edition of Bartlett's Quotations, which he gave to me, a book which has given me many hours of pleasurable browsing and would be the first to be saved in case of a fire. It served me well when I needed an apt quote for a college paper or, on becoming a superintendent of schools, for a speech, serving my children as well and, in view of its history, continuing to be of special value to me.

It was at this time that I was awarded the Bronze Star and, of particular value, my combat infantrymans' badge for combat service. The regular infantry badge was a rifle on a rectangular blue background, while the combat badge had an added circular wreath, making it a very proud possession of all infantrymen who had served under fire.

\section{TBD}
Barking back to when I was in Metz and required to label and turn in all superfluous personal items before being transported north to the Battle of the Bulge, these stored items were returned to me just before returning stateside, including my Brownie camera which contained an expended film. Upon developing I found several pictures of French girls in show-off poses, these within a warehouse setting where, obviously, these goods were stored. Their efforts produced a bit of unexpected entertainment, as was intended.

\section{TBD}
A disturbing event involved Little Joe and a Lieutenant Griswold, one of our officers who had lost an eye in the Italian campaign. Though Little Joe was a French citizen he had an Italian name and family background, which was, apparently, noted by the Lieutenant. The Sergeant who had previously cared for Little Joe had an Army uniform cut down to fit him, adorning it with all the same stripes, hash marks, and unit designations contained on his uniform. Unknown to me, this little Italian boy wearing the decorated Army uniform was very offensive to the Lieutenant, no doubt as a result of his combat duty against the Italians, causing him to stop Little Joe as he was going through the chow line, calling for a sharp knife and, in the presence of all the men in the hall, proceeded to strip Little Joe of every adornment sewed on his uniform. Seeing my obvious agitation, the Lieutenant instructed me to ``Stay out of this, Corporal'' which I was forced to do under Army rules. To Little Joe's credit, he stood there like a man, tears running down his eyes, suffering gross humiliation. No wonder GI's like myself had little respect for certain officers.

Throughout those months, as in all which came before, I continued to receive letters from home from my most faithful correspondents, weekly letters from my father, almost weekly notes from my Godmother, Auntie Claire, and from Mary Sullivan, a teacher who lived in the Highlands and took it upon herself to write regularly to a number of her former students. The importance of mail from home to a GI cannot be understood fully unless one has had a similar experience. How eager we would be to hear our name called at mail call.

As spring went along, I began to get very excited about my coming return to the states for discharge, finally receiving my travel orders directing me to report to the port of Marseille for surface transportation back home. What an exciting prospect! My year in southern France was a great experience, one which whetted my appetite for returning with my family in the years which followed. Aix-en-Provence, Le Baux, Marseille, Avignon, Arles, and many other locations all became very familiar to Rita and me, even to have the fun of visiting once again with Odette Rougier, who worked with me in Aix. I shall ever be grateful for those great days\ldots\ a time of travel, adventure, and wonderment for me.
\end{document}
