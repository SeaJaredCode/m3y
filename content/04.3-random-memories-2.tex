\documentclass[../m3y]{subfiles}
\externaldocument{../m3y}
% \graphicspath{{\subfix{../images/}}}
\begin{document}
\section{TBD}
The-Army provided transportation to and from Marseille, this on large open semi-trailers lined with benches, and the troops queued up every day for a few hours in Marseille, respite from the days when hell would re-commence in a few weeks. We got to know Marseille, an ancient city of over a million, quite well with our visits in town centering around Rue de Cannibierre, Marseille's main thoroughfare which ran right down to the old port, which was the center for the extensive black marketing activities, France having become a cigarette and chocolate bar economy. Some fifty-four years later my wife, Rita, and I got to spend several days in a hotel overlooking this very spot, with the famous Pont de Lourdes church atop a steep hill across the bay. There was a large Red Cross club at the far end of the street, across from a large church, and the Army had taken over two movie theaters along the Rue de Cannibierre. Outside of each there were small stalls selling two kinds of sandwiches, one of some small fish (which I never tried), the other a sandwich of fresh tomatoes and sweet onions between great crusty bread, these wrapped in newspaper sheets. It resulted in a theater full of GI's munching while they watched an American movie.

Another treat was the delicious large black heart cherries that were sold in the marketplace, which were as delicious as could be desired, causing me to eat to the point of being uncomfortable.

The Army maintained an outdoor kitchen and mess at our semi-trailer drop-off/pick-up point, not far from the Rue de Cannebierre. After eating we would go through a line to dump our garbage and wash our mess kits in boiling water. French women and children would cluster at the beginning of the line, encouraging us to dump our unfinished food into the GI gallon fruit cans they held on improvised string handles. This meant everything went in together\ldots\ beet juice, bread crusts, meat gristle, fruit cocktail juice, etc\@. Pretty unsavory, but a blessing to those with nothing.

\section{TBD}
With cigarettes, bought in the PX for \$1.20 a carton, bringing \$20 on the black market, and a five cent Hershey bar bringing a dollar or two, most all GI's took to selling some of their purchases to augment their meager salaries. It was at this time that I took up smoking cigars, which were not rationed, in order to sell my cigarettes. I remember well puffing my first cigar riding on a semi into Marseille, becoming sick as a dog. For some reason, the image of this skinny kid smoking a big stogie always brought hilarious laughter from Rita and kids when I would recount the tale in later years. I don't think I looked \emph{that} funny.

\section{TBD}
As mentioned previously, as long as we performed our duties responsibly we were left pretty much alone, freed of the usual Army folderol. This atmosphere didn't extend to reporting late for work, however, which Tony Germano and I had done a few times due to his being a very slow starter in the morning. This led to a really funny event in that our new captain, not nearly as easy-going as the previous major, angrily sent us back to our tent area with the threat of a court-martial, a ridiculous possibility. Our own company commander, who was responsible for deciding our fate, chuckled when telling us how very angry the captain was when he telephoned to press his charge, it being obvious that he, too, was not very fond of our office head. He said something like, ``How'd you boys like a little company punishment? The next two days are to be spent shoveling on the company coal pile'' an assignment which was pretty funny in that all we could do was to shovel the coal into a new pile and then shovel it back to where it was before\ldots\ all done with anything but haste in the hot sun. We took a lot of razzing from our buddies going past the coal piles, giving us a few laughs as well. Even got a wry smile from our company commander when he greeted us with, ``How you guys doin'?''

\section{TBD}
It was at this time that I had the opportunity to expand my cultural horizons, despite being a lowly Gl living in a tent on a windswept plain. Some of my friends, having come from more privileged backgrounds, were quite familiar with classical music, and I was soon accompanying them to attend operas and symphony concerts in Marseille, a new experience that I found enjoyable. My musical interest being thus revived after being dormant since joining the Army, I took the further step of enrolling in a special program wherein the Army paid for private piano lessons with a fine French teacher, Madame Chouchard, whose husband was a member of the Marseille Symphony Orchestra. This was, apparently, part of a program to aid the local community, which was still suffering from severe shortages of every sort.

Though I had weekly lessons, my progress was somewhat limited in that the only piano on base required close to an hour's walk each way for practice purposes. Despite this limitation on my progress. Madame Chouchard was very happy to have me as a pupil as, in addition to providing some income, I usually brought some staples from our Army mess, small amounts of sugar, coffee, or marmalade. Though they were finding their way out of a difficult period, they lived in a very nice apartment, reflecting a cultured atmosphere, and my lessons soon expanded into social visits as well.

In this way I learned that the Germans were not above corruption when they were occupying the city. The Chouchard's son, in his twenties, was involved in distributing an underground newspaper, a very serious offense which led to his arrest by the Gestapo, subsequent ``trial'', and sentence to death. Drawing upon the financial resources of the family, the Chouchards were able to bribe certain German officials gaining, ultimately, the quiet release of their son. Another example that, despite ideology, hard cash can sometimes influence the course of serious events.

\section{TBD}
As our meals were eaten outdoors on picnic style tables, we always had a few stray dogs running about, ready to pounce on any scraps. One cream colored dog, about the size of a small poodle, attached himself to me and was soon sleeping on my cot at night and accompanying me to the office. He was smart, responsive to my training and attention, becoming a favorite of several of my friends, becoming my pet for several months. This was to end abruptly on Christmas Eve, 1945, when Tony Germano and I went in to attend Midnight Mass in Marseille, returning very late to find that my dog was not waiting in our tent as expected. The next hour was spent scouting around with flashlights to see if the dog might have been hit by an Army truck. His disappearance was really puzzling, as he was accustomed to his freedom around the company area, always returning to me and the warmth of our tent. This mystery was cleared up to some extent four months later, at carnival time in Aix-en-Provence, a tale which will be told at a later point. Interesting, though, that each ofmy two Christmases overseas were connected by events involving dogs. (See Appendix)

\section{TBD}
V-J Day, marking the complete end of World War\,II, occurred the second week in August, celebrated at Calas Staging Area by the firing of weapons, even a few grenades, that the men had accumulated as personal possessions… really a lot of shooting in the air. As there were some injuries, this led to a massive inspection of each man's foot locker the next day, however, with a large number of weapons, many German souvenirs, being confiscated. Actually there was a lot of, “liberated” stuff around our camp as the temporary units passing through were given stern warnings about not taking anything of that sort aboard ship, resulting in their peddling the stuff to us and to French civilians. In this way I acquired a fine pair or Army binoculars and a small German accordion, which I taught myself to play, providing a bit of jolly music at family gatherings in later years.

V-J Day also resulted in the eventual closing of the three staging areas serving the port. As there was no need to ship units to the Pacific, the Army started redeploying all ofthe troops home through ports in the North, particularly through Le Havre, where thousands of men were processed through Camp Lucky Strike. Our tents weren't folded, however, until the early part of 1946.

\section{TBD}
When the weather got cold in October we were overrun with field mice, impossible to keep out of our tents. Finally, the Army issued a mousetrap to each man, with Tony Germano and I placing a bet on who could catch the most mice that night. Soon after lights out my trap went off, giving me the task of climbing out of my cozy sleeping bag to bring the trap outside, where I disposed of the mouse in a barrel. I put the trap, which still had some bacon bait left back under my cot. Soon after getting comfortable in my sleeping bag, the trap went off again, causing me to go through the same disposal act as before. I started gloating over the success of my trap, as Tony had caught nothing. I'm ashamed to confess that it took me several mouse-catching episodes later to wise up to the tact that Tony was having a good laugh for his \$1 bet, as he had not set his trap, leaving the exterminator's job to me. He kidded me again about my prowess as a mouse catcher when I visited him in Chicago after the war.

\section{TBD}
My stay at Calas Staging Area also provided some stunning opportunities for traveling in Europe, which I took advantage of at every tum. This included an early trip to Avignon with Tony Germano, with the initial intent to call on a Greek American from Lowell running a hotel, having married and remained in France after the First World War. Another GI friend had stayed at this hotel and told me about the owner upon returning, saying the owner looked forward to meeting me.

This turned out to be a very special weekend. The hotel owner from Lowell would not let us pay for our rooms or meals at the hotel but, to my special pleasure, Tony and I got to spend hours roaming through the Palace of the Popes, a massive fortress where dual popes lived for forty years, challenging the supremacy of the pope in Rome. This was before the return of tourism, of course, and we had the free run of the place, finding massive kitchens, grand receiving rooms, dungeons, extensive ramparts, etc.\@ with great views of the fast-moving river, partially straddled by the ``Pont de Avignon'' celebrated in song by every French 1 student, the centuries-old streets, and the wall surrounding it all.
World War\thinspace\ II\@.

In the months that followed, I got two additional travel vacations, getting a week's leave in each case. The first was to Nice, on the French Riviera, the second to Switzerland. Each was to provide rich experiences which ignited my smoldering travel ember, bringing it to a flame that continues to this day.

France, itself, was still in very rough shape economically, physically, and psychologically, having suffered the blows wrought by the German occupation and defense, to be followed by the mighty Allied military, wreaking havoc wherever they found German resistance. Food and other essentials were strictly rationed, this facilitating a thriving black market in cigarettes, chocolate, blankets, and anything else some enterprising soldiers would find to sell. In 1945 there was no tourism, of course, which made it possible for the Army to negotiate taking over the entire hotel facilities in Nice, even the Casino, which became a Red Cross club, designating the entire city as an R\&R facility for enlisted men. Similarly, Cannes, just a few miles away, was taken over for the officers, consistant with the strong Army caste system.

My week's leave in Nice came to an abrupt halt when I developed a very high fever the first night there. My roommate said that I behaved deliriously during the night, causing him to report my condition to the desk without consulting me. Next thing I knew on the following morning was the arrival of two Army medics who, after making a preliminary assessment of my condition, placed me on a stretcher and, in short order, I was once again a patient in a military hospital.

This stay, which lasted almost two weeks, was really a delight. The hospital had served as a sanitarium for the rich before the war, consisting of stunning buildings and a great tropical garden, statues and all, overlooking the Mediterranean. Apparently, I had contacted some kind of bug which responded to the sulfa drugs provided, so that within days I was feeling fine, having the run of the place, including a well stacked Special Services library, spending hours reading amid the luxuriant tropical trees and plantings in the garden. The Army, wisely perhaps, kept me longer than I felt I needed, but there was no complaining from me. The nice part was that they cut orders permitting me to enjoy my initial 7 days in assigning me to the Negresco Hotel, Nice's finest facility on the Rue de Anglais, overlooking the handsome crescent along Nice's waterfront.

I got special pleasure in being given a small private room all to myself; my first in two years of soldiering, relishing the sense of privacy that that bedroom provided. Other memories relate to eating in the Negresco's main dining with every kind of silver piece on the table --- producing a lot of GI speculation as to the purpose of each --- the services of waiters in tuxedos, plus live dinner music. Best of all was what the French chefs were able to create out of standard Army rations, probably serving the best meals available in France at that time. The Army had wisely contracted to utilize experienced hotel staffs as a boost to the locals, permitting us to taste a slice of life far beyond anything experienced in our civilian days.

I also took advantage of the tours available, seeing the old part of the city, the open markets, and an all day trip to Grasse to see how perfume was made from the petals of flowers. Another trip took us to Monaco, touring the Casino, and seeing the spectacular Riviera scenery, traveling on the Grand Corniche, parallel to the ocean, All these sights were to be revisited with the family in the years that followed.

\section{TBD}
I found my trip to Switzerland over Thanksgiving 1945 almost overwhelming in its sights and delights. As is well known, Switzerland remained neutral throughout the war, escaping the horrors experienced by all of her close neighbors. To cross into a country with its towns and cities intact, its shops full of every kind of goody, the people well dressed, and all of this against a stunning alpine background, was truly awesome… and I made the most of it. All the touring was done on the excellent Swiss train system, and I saw some of the best that Switzerland had to offer. First stop was Geneva, with a tour of the League of Nations headquarters, followed by two days in Montreux, a resort city on the lake surrounded by towering Alpine peaks.

While there I did some shopping for family Christmas presents (my mother kept the music box I sent her on her bureau until her dying day), but the tour of the Castle of Chillon, followed by a cog railway trip up the Roche de Naye, were the highlights of the entire trip, in my view. The view from the top of the mountain (probably 13,000 or 14,000 ft.) seemed like the top of the world, as we could see Alpine peaks located in Bavaria, Italy, France, and Austria, all covered in white snow and glaciers, jagged massifs poking through. I still have a lot of the black and white photographs I took that day.

I meant to mention that my traveling companion on that trip was Charlie Monson, a practicing Mormon, but not at all stuffy as a roommate. When we were in Geneva we lucked out in being given an opulent suite in the Grand Hotel, no doubt occupied by diplomats before the war. Talk about living.

Berne, Switzerland's capital was also included on the itinerary, where we saw the well known bear pit and the amazing tower clock, still running after some number of centuries. Berne was memorable in another, more personal, way in that, having placed a phone call home several days before, I was given an appointed time when I could talk to my parents via Atlantic cable. We were limited to three minutes\ldots\@. I can't recall any details of our discussion, but it surely was a wonderful few minutes for this GI who, like the rest of the Army, yearned to see his family again.

It was in Berne, also, that Charlie Monson and I accepted an invitation to dinner with a Swiss family, nice folks who spoke English and provided a warm introduction to Swiss life. He was a botanist, and much of the evening was spent in seeing his slides of alpine wildflowers. Though I've forgotten their names, I still remember their address of \#7 Cyrostra{\ss}e, a recollection that enabled me to re-visit them in later years, introducing them to Rita and our children.

It occurs to me that, despite my contacts with the Chouchards in Marseille, my only dinner invitations were from English and Swiss folks, which is rather symbolic of the French attitude toward Americans.

These travel experiences, plus some which followed, did much, I believe, in whetting my postwar appetite for subsequent European employment and travel with my family in the
1953--63 period. I had been bitten but good.
\end{document}
