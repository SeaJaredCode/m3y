\documentclass[../m3y]{subfiles}
\externaldocument{../m3y}
% \graphicspath{{\subfix{../images/}}}
\begin{document}
I have found it interesting to read in books on WW\thinspace II history how the Army had suffered such excessive losses of infantrymen at that point in time that they found it necessary to have each non-infantry (service and headquarters) unit to re-assign 10\% of its personnel to combat infantry units, some 65,000 men being transferred in late October and November. Understandably, that 10\% included the very poorest soldiers (the so-called f-ups), which unit commanders were glad to get rid of. It also included one Bernard Huntley, who not only had one, but two infantry training stints on his record, plus being the newest and least essential kid on the artillery block. Thus ended my very short career with a big gun outfit. Within days I was a new replacement in Co.\@ B, \nth{2} Battalion, \nth{101} Infantry Regiment, of the \nth{26} (Yankee) Division, an outfit well known in New England, and now a part of General George Patton's Third Army, which was in the process of capturing St.\@ D'Arc, the last stronghold of the fortress city of Metz. I was one of many replacements comprising most of our all extremely unhappy with their new assignment, quickly learning what the M-1 rifle was all about, as well as other aspects of life as an infantryman.

This was in the early weeks of December, and our time was spent mainly in field problems, where the non-coms would walk us through some of the actual skirmishes that had occurred just days earlier. Most of the new arrivals, having come from service units, were familiar with the carbine rifle, a smaller and lighter weapon than the M-1 rifle used in the infantry. I, along with infantry trained others, was given the task of teaching the men how to break down and care for their M-1 rifle, close to the best friend they would have in combat.

While many of the new transferred replacements represented the bottom of the barrel, I was able to connect with a few guys with some intelligence, and I developed a respect for two of our non-coms, a Sergeant Lahue who, amazingly, was from Lowell, and a real American Indian who was our corporal. They were both experienced combat soldiers, holding the unit together really, for our Lieutenant and Captain, also transferees from service units, lacking infantry training, did not display much that would inspire our confidence in their leadership. This assessment was confirmed later when our captain went to pieces under the strain of combat and had to be sent to the rear.

It was easy to distinguish the regular experienced combat infantrymen of the \nth{26} Division, as they were really happy and relaxed to find themselves in the rear for a while, not displaying much glee but obviously making the most of getting hot chow and sleeping inside a building, even though it was unheated and there were no bunks.

This rest and re-organization period came to an abrupt halt upon returning from field training on the \nth{19} of December, we were told to pack up our things and assemble outside of the barracks after we had our supper. Personal effects other than what one could carry on his person were put in separate duffel bags, tagged, and turned in for storage, recovered, in my case, over a year later. We were then all loaded into trucks and were soon barreling down the road, snuggling against each other to reduce the cold December air, with our destination unspecified

We rode north for several hours, eventually being dropped off someplace in Luxembourg. In the morning we were assembled and shown a map which outlined the depth of the German penetration through our lines and told how we would be pressing north to engage the enemy which, by that point had gone far west beyond where we would be meeting them. After a hike of some 12 -- 14 miles, we dug in and awaited our baptism of fire, which occurred the next day, and all the wondering about what it would be like to be a rifleman in actual combat was soon a thing of the past.
\end{document}
