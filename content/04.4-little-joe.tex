\documentclass[../m3y]{subfiles}
\externaldocument{../m3y}
\begin{document}
I first met Little Joe (Pascal Zenier) while at Calas Staging Area, he being one of several wartime waifs who, under usually dire circumstances, attached themselves to various combat units. While not an official policy, the Army seemed tolerant of these arrangements, and the American GI was typically generous and sometimes caring, particularly toward children made orphans through wartime actions. Maybe, too, these kids reminded soldiers of their own children or nephews, providing a bit of ``family'' atmosphere within a rigid military culture.

Little Joe befriended a staff sergeant, apparently a kindly and responsible individual who, upon being redeployed to the States through Marseille, arranged for the boy to be cared for by a French woman until such time as he could arrange for his adoption and subsequent move to the States. It was obvious from the general demeanor and manners of the boy that the sergeant had invested a lot of time and care in his development, his only bad habits being smoking (he was eleven years old) and the occasional use of the four letter words found within GI society. He was very bright, having acquired a fair command of English, fastidious in his personal appearance, and a very engaging young man, at ease within the military setting. His father, a soldier, was killed in the war, while his mother was a casualty of the fighting in their village.

The woman charged with his care, for which she was paid some amount by the sergeant, turned him out on the streets as soon as the sergeant was shipped out, resulting in Joe wisely finding his way to our military post.

There came a day when I, along with two of my tentmates, came across Joe as we returned from lunch, finding him weeping heavily. It seems that, while he had been accepted by one set of tentmates, they were being dispersed, as we all were, to other points due to the coming shutdown of the whole staging area, and no one was coming forth to assume any responsibility for him. We let him bring his stuff over to our tent while we considered  what to do next. Tony Germano and I wound up with the main temporary responsibility for Joe for the next weeks until the camp closed. At that point Tony and I were assigned to different locations, with Little Joe's welfare falling in my lap, and I subsequently took him to my new assignment in Aix-en-Provence, joining the room I shared with Bill Martin (more about Bill and Aix later).

\begin{figure}[h]
\centering
\includegraphics{little-joe-me-tony}\\
\medskip
{\newtimes\textsc{Little Joe, Me and Tony Germano at La Rotonde, Aix}}
\end{figure}

In the following months, I sent several letters to the sergeant who first cared for Joe who, it turned out, was married but childless, living in Missouri. All efforts to contact him were fruitless, and I concluded that he, or maybe his wife, felt different about resuming any responsibility for Joe after a period of time. (I did reach his wife by phone years later after the sergeant had died, and she claimed to know little about her husband's commitments to Joe).

During this period, Joe became the somewhat pampered mascot of our unit in Aix. He hung around the area, in our room or the Red Cross Club, while I worked, joining with us for meals and accompanying us on trips to Marseille or other places on our time off. I was hopeful, of course, to get him on his way with the sergeant before I myself was returned to the States. This became a bit sticky as my final weeks were looming in sight, our unit itself was breaking up, and the new officers in charge, not having the kind of understanding developed by combat duty, began putting pressure on me to get Little Joe out of the area.


\begin{figure}[h]
\centering
\includegraphics{little-joe-2}\\
\medskip
{\newtimes\textsc{Little Joe, La Rotonde, Aix}}
\end{figure}
This led to my decision to place him in an orphanage but, upon breaking the news to Joe, he began crying and admitted that he thought he still had family back in Jarny, a coal mining town near Metz. In further consultation with our commanding officer, I was given appropriate written orders to permit my bringing him to Jarny to see what arrangements might be possible. This was a two day trip, a sad one for both of us, with an overnight stop at a hotel in Dijon, partly traveling by train, the final leg being a jeep, with a driver provided by the post in Metz, a city where I experienced the beginning of my experience in combat many months before.

Little Joe guided us to a house where we found one of his aunts, and I learned that he had an uncle and a brother, both of whom were working in the mine when I arrived. As the loan of the jeep and driver required my prompt return, I sadly left Little Joe in the care of his aunt, who assured me that he would be welcomed back in her household. The GI driving the jeep was kind enough to ignore the tears that couldn't be restrained on our return to Metz.

I was glad, though, that Joe seemed pleased to be returned to his family, as well as they to see him, and thus one young man's rather amazing wartime adventure with the U.S. Army came to a close.

\begin{figure}[h]
\centering
\includegraphics{little-joe-3}\\
\medskip
{\newtimes\textsc{Little Joe in Aix}}
\end{figure}
There is an epilogue to this story. I maintained sporadic contact with Little Joe in the years that followed, receiving letters in English written by his school teacher, always with the teacher requesting classroom supplies at the same time. Over time I sent a few boxes, but the letters petered out as life went on for both of us until, that is, I was able, along with Rita and our children, to have a reunion with Joe some ten years later upon my being employed as a principal in several schools for dependent Army children in Germany. He was living with his married brother's family in Jarny, where I had brought him years before, and employed as a coal miner. We arrived unannounced, finding him at work deep beneath the earth. I was directed to the manager's office at the mine where he made a phone connection with Joe, far below us. His response was warm indeed, obviously he was amazed to find me in his village and, after the hour or more that it took him to be brought from below, we had a very special reunion.

We visited Joe and his family several times thereafter, with Rita usually bringing clothes that our girls had which were welcomed as they were living under minimal conditions. Later I was transferred to the other side of Germany, and Joe took a new position in a factory in Chamonix, France, where we had our final visit on one of our winter holidays. He was still single and somewhat well-known in amateur soccer circles. We always found him very warm and glad to see us, he doted on our children, and it was good to find that he was making his way successfully in the world.

Going back to our first reunion, I was very touched when, upon our leaving, he put his arm around me and, looking at me through tearful eyes, repeated ``Thanks, thank you \ldots\ thank you'' referring to that time when I played a vital role in his young life. Little Joe was now a man and had an appreciation that he may have lacked as a young lad long ago, and I realized more fully that this was one of the better chapters in my own life.

\begin{figure}[h]
\centering
\includegraphics{little-joe}\\
\medskip
{\newtimes\textsc{Little Joe (Pascal Zenier), 1946}}
\end{figure}
\end{document}
