\documentclass[../m3y]{subfiles}
\externaldocument{../m3y}
\begin{document}

\begin{figure}
\centering
\includegraphics{ft-benning}\\
\medskip
{\newtimes\textsc{Paul Love + Me, Ft.\thinspace{}Benning, 1944}}
\end{figure}

Paul Love, who has also made the cut for radio school, and I both found school life much to our liking, standing out in stark contrast to our physically demanding basic training. Fort Benning, Georgia, is the site of the Infantry School and is the best permanent post in that branch of the service. Post buildings and barracks were of red brick, surrounded by well-manicured lawns and many shade trees, and there was a large residential area that was as nice or better than most superior suburbs in civilian life. Recreational facilities were excellent, with a handsome theater, post library, and several swimming pools. The officers had their own clubs and pools, of course, while the pools for the enlisted men were segregated by color, as desegregation of the Army was still waiting for the courageous actions of Harry Truman in the late forties.

We were privileged in other ways, being relieved of having to care for a rifle and full field pack, and having only one brief period of daily calisthenics. Our training focused on more advanced radio procedures, continued Morse code training, and the use of more sophisticated equipment. We operated on two shifts, 6 am to 2 pm, and 2 pm to 10 pm, spending two weeks on each shift. Having free time in the afternoon was great, and we often spent it in the swimming pool or the library. Some evenings Paul and I would walk around the residential area which, with kids playing around, was as far away as we could get from the regimented Army atmosphere.

As always, I did a lot of reading, remembering many afternoon hours of bunking out when I was on the 6 am to 2 pm shift, or, another favorite pastime, watching the airborne troops floating slowly to the ground from their parachute towers, which were close by. Another strong recollection is of the very hot and humid weather. Flat roof areas were spaced out on the third floor of our barracks, and these became our alternate sleeping areas when the barracks were suffocating. About a half dozen of us would lug our mattresses up to the roof, spending many hours chatting under the stars and enjoying the cooler night air.

Upon graduating I was given an MOS (military occupational specialty) number designating me as a communications specialist, valued in particular in that it superseded the common infantryman designation I had upon finishing basic training, thus I might anticipate any future combat action to be somewhat to the rear of the front lines. As will be seen, this expectation was short-lived.

Following this training I was given a furlough back to Massachusetts, which was greatly anticipated and enjoyed. I think my parents were distressed that I hadn't put on but a few pounds beyond my regular 125. As I look at old pictures now, I sure didn't resemble any ferocious combat-ready soldier, despite my training. It was good to be home again, however. To my puzzlement, I became one of several Communications School graduates who were reassigned to commence another infantry basic training program in Camp Van Dom, Mississippi. \emph{Another} BASIC? In \emph{Mississippi}? In \emph{July}? Geez!!!

\begin{figure}[p]
\centering
\includegraphics{ft-benning-2}\\
\medskip
{\newtimes\textsc{Me, Ft.\thinspace{}Benning, 1944}}
\end{figure}

\end{document}
