\documentclass[../m3y]{subfiles}
\externaldocument{../m3y}
\begin{document}
My return sailing through the Mediterranean and across the Atlantic to Baltimore could not have contrasted more with my crossing in the other direction a year and a half before. Instead of returning on a troopship, about fifty of us found ourselves in a large space on what, we learned, is called a ``sea train'' in that it is especially designed to accommodate rolling rail stock, with tracks within the hold area. In our case, the ship held a number of Large diesel engines which we were told were being returned to the United States because of an error in sending over engines with the wrong rail size for European tracks. While this seemed far-fetched, the evidence was there before our eyes as, surely, new engines were not being manufactured in Europe at that time for use in the States. One supposes that in the large scheme of things it is possible for some lower level bureaucrat to make the wrong selection on some complicated shipping document.

In any event, it all worked to our advantage in that we were fed Navy chow, the April weather heading south across the Atlantic was stunning and, with so few of us on board, there was little to do but take a towel and sun ourselves on the deck, wearing only underwear, and reading a good book. I recall observing that the sea was so lake-like that the ship hardly dipped as we crossed through the placid waters.

On the first day out I responded to a call for someone to put out the daily ship's newspaper. This called for an early visit to the bridge to obtain the figures telling us how much progress had been made in the past 24 hours, then to the radio room where I was given a few paragraphs of world news that had been received overnight. Armed with this, plus some canned items made available to me, I would then proceed to type the daily news, usually just a four-page document, which would then be mimeographed and distributed.

While it was fun to play at being an editor in this small way, I got myself in some mild difficulty when I printed an item of my own origin, something I had seen in a magazine some time before. It went like this:

\begin{quote}
\emph{Oh, John, let's not park here.\\
Oh, John, let's not park.\\
Oh, John, let's not.\\
Oh, John, let's!\\
Oh, John!!\\
Oh-h-h-h!!!!!!
}
\end{quote}

I guess that the Captain was not a typical navy man, as he had the bridge officer inform me that there would be no more such stuff in the ship's newspaper. ``Ay, ay, Sir, no more.''

I don't recall how long we were at sea on this somewhat longer ocean route but, in due time, we sailed into Baltimore harbor on one hot and humid afternoon, though the weather was of no concern to us as we lined the rail, relishing the glimpses of American girls, automobiles, signs in English, and the sight of hard American soil. Broad grins abounded.

After a few days of processing, I then traveled by rail to Ft.\@ Devens, site of my original induction, telephoning my parents planning for their picking me up on the following day, when I would, once again, become a civilian \ldots\ free of military authority and in charge of my own destiny.

It's rather interesting to note that I had sailed out of New York harbor on October \nth{22} about eighteen months before, on my mother's birthday and now, on May \nth{6}, I was being welcomed home on Mother's Day. The whole family drove up to Ft.\@ Devens on that very, very special day. Home safe, the entire family reunited (Jim having been discharged earlier), I returned to the comfort and security of 539 School Street, thrilled in seeing all that was so very familiar, changed in many ways from the eighteen-year-old who left three years before but happy in anticipation of the life ahead.

The days that followed brought happiness and personal success far beyond what was expected back in May of 1946. Underneath and through the decades, however, this one-time foot soldier has never forgotten what life was like in those bitter woods of Luxembourg \ldots\ nor failed to appreciate the luck of it all.

\begin{flushright}St.\@ Thomas, USVI, 2004\end{flushright}
\end{document}
