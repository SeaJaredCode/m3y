\documentclass[../m3y]{subfiles}
\externaldocument{../m3y}
\begin{document}
My recollections of life aboard a troopship at the height of the war may seem as few, this because the experience, from an enlisted man's point of view, at least, is one of utter boredom, consisting of three activities in the main, foremost of which is standing in lines most of the day, working toward the mess hall for each of your three meals.

While most likely to be coincidental, the lengthy procedure in trying to feed several thousand men under limited and crowded conditions seemed deliberately designed to occupy the most substantial part of one's waking hours. The lines would slowly move up one stairway (or ladder, as the Navy would say), out on deck, then reenter another stairwell, going down into the lower labyrinths of the ship, only to repeat the process at a different part of the ship, and finally to approach the mess hall, announcing itself by the stale odor of steamy food and many bodies. One presented his personal mess gear to the assembled line of servers but, unexpectedly, the mess hall had no chairs or tables, rather numerous stations where one stood while eating from the counter. One could readily understand that when it comes to troopships, every square foot saved meant that more men could to transported, a vital consideration at that time.

The second predominant memory is life in the bowels of the ship, where vast holds were full of narrow bunks (canvas pallets) five tiers high, upon which we stored our full infantryman's share of equipment and one's limited possessions. I felt fortunate, indeed, to be assigned to a top bunk, as that meant that I could read a book without having to extend my arms into the aisle as those below found necessary in order to have light. There were no portholes and, though stuffy, it was comfortable enough to sack out, with one's book reading underscored by the constant sound of the sea on the other side of our metal shell. During rough weather, the stern would lift sufficiently high out of the water to free the prop (or props), which would send a mighty shudder throughout the hold.

\begin{figure}
\centering
\includegraphics[width=0.8\textwidth]{life-aboard-troopship}\\
\medskip
{\newtimes\textsc{Life aboard a troopship}}
\end{figure}

I was most fortunate in having picked up a pocket-book edition (as paperbacks were called then) of Shaw's ``Lust For Life'', a biography of Vincent Van Gogh. I found it completely captivating, enabling me to be transported mentally and emotionally to \nth{19} century Holland and, then, to Arles in southern France, a location I got to know very well in my future years.

The third aspect I recall of shipboard life was spending time on deck when the weather permitted, where one felt lucky to capture sufficient space to sit on the deck, and even more fortunate if you had a wall or funnel to lean against. The very vastness of the sea is especially impressive on one's first Atlantic crossing, evocative of long thoughts about what might be ahead, wondering if a return journey would be mine one day, thinking about family, missing the stateside routine of daily mail call (how important and precious it was to receive letters from home, and thrilling to get a package, even though it had to be shared with others), wondering if there were any German U-Boats in our corner of the sea, and similar thoughts.

While we could see only a limited number of ships that made up our convoy, there was some real excitement one day when a distant ship, most likely a destroyer, dropped a depth bomb in response to something that concerned them. We had no further information of what may have provoked or resulted from this event, for all we knew it could have been some sort of drill, but it caused quite a buzz throughout the whole ship.

Our training on-board consisted of calisthenics and, to our pleasant surprise, group lectures on how to be a good guest of Great Britain \ldots\ hey!\@ we would get some months in England! Visions of cozy pubs, friendly girls, downtown London, movie theaters, etc.\@ flashed through our minds while we were hearing about pence and pounds, and how to behave in a proper fashion. Our anticipation really grew once we had this indication of a coming reprieve from possible combat duty.
\end{document}
