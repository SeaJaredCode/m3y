\documentclass[../m3y]{subfiles}
\externaldocument{../m3y}
% \graphicspath{{\subfix{../images/}}}
\begin{document}
My first real awareness that I may wind up in the service came on a cold December afternoon in Lowell, where I, with some of my family, enjoyed a great concert of folk music by the Trapp family, newly arrived from Austria, with their ``Sound of Music'' musical and movie fame still years ahead of them. Upon leaving the auditorium we were bombarded by Lowell Sun newsboys shouting ``Extra, extra, Japanese bomb us at Pearl Harbor \ldots\ read all about it,'' etc. We were both electrified and astonished upon learning the details of that sneak attack. On Monday morning the principal of Lowell High School, where I was a senior, piped President Roosevelt's declaration of war into each classroom, a momentous action that soon brought millions of men, and a few women, into the armed services, and which would significantly change the daily lives of every American for years to come.


The first change affecting me directly was my selection of several hastily added choices in our high school curriculum\ldots\ power plants (all about airplane engines), Morse radio code, and meteorology reflected my anticipation of coming military service. Our June, 1943 graduation ceremony placed a heavy focus on patriotism in the speeches and music (``This Is My Country'' etc.), and, no doubt, we all did a bit of wondering about what the future would bring to each of us.


I don't recall much about that summer. I went to work full time in the family hardware business, my father having lost his single non-family employee and my older brother, Jimmy, to the services. Shortly after my \nth{18} birthday in August, I received my selectIve service notIce to report to Ft. Devens for my physical. By October, I was on my way to basic infantry training in Camp Croft, just outside of Spartanburg, South Carolina.
\end{document}