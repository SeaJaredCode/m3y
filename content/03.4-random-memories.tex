\documentclass[../m3y]{subfiles}
\externaldocument{../m3y}
% \graphicspath{{\subfix{../images/}}}
\begin{document}
\section{A Matter of Proximity}
One day a few of us were charged with bringing back a group of 40--50 captured Germans to the battalion collection point for POW's. They were docile enough, in the same tough shape as we were, but it was a bit hazardous as there was always the possibility of being picked off by a German sniper, essentially suicide soldiers with special rifle sights who would stay hidden, most often high in a tree, an area where the regular German forces had been routed. The sniper's job was to wreak whatever havoc he could before being picked off himself.

We, therefore, had reason to keep prodding the group to keep moving rapidly. One of our men got impatient, or worried enough, that he started firing at the ranks of the prisoners, hitting one in the buttocks. Though we gave him hell for carelessness, nothing more was done, no reporting the incident, etc.\@ By this point, I guess we had all become sufficiently weary and desensitized by what we were that his behavior didn't produce the kind of outrage and formal response that it would have in a different setting.

Incidentally, as we were reaching the battalion collection point we were filmed by one of the cameramen covering the war. I've always hoped that someday I would see a few seconds of that event in subsequent documentaries or photography collections. No luck.

I got a real unexpected bonus that day, however, as it became dark before we could return to our unit, so we joined some tankers who were holed up in a barn, enabling us to get a super night's rest without interruption. In reality, we were not very far behind the front line, much firing was still occurring, but the knowledge that someone else was between you and the Germans gave us welcome relief. It was all a matter of proximity.


\section{A Humble Comfort}
On another night I found myself with a small group allowed a night's sleeping break in the cellar of a farmhouse. I missed out on getting a spot on the sugar beet, possibly large turnip, storage pile, settling for a skinny board, probably 12--15 inches wide, between me and the floor. I don't recall how, but we came into possession of a few more blankets than usual, giving me the luxury of sleeping under two blankets. Other men arrived as time went on and at some point during the night I woke up shivering, with one blanket less. Don't know where it went, but someone was more comfortable than he would have been otherwise.

Later, studying Shakespeare's King Lear in college I read something about how deprivation can make even a humble comfort seem like much more, putting me in mind of how glad I was to find that skinny board. Tolstoy also made a reference in Anna Karenina:

\begin{quote}
    \emph{There are no conditions to which a man cannot get accustomed, especially if he sees that everyone around him lives in the same way. Levin would not have believed it possible three months earlier that he could go quietly to sleep in the circumstances he now found himself.}
\end{quote}

\section{My Lucky Day}
My closest call to becoming a casualty from enemy fire occurred one day when we were sitting up in our foxholes, with no particular combat activity going on in our vicinity. We heard the noise made when nearby German mortar shells were dropped into their firing tubes, this producing the immediate response of ducking down into foxhole before the shells made their arc and landed in our area. Simultaneously, there was quite a lot of rifle and machine-gun fire thrown our way. My arms holding my rifle went up as I threw myself down on my back, and a bullet grazed my right arm, not doing any serious harm. Had we not responded to the first sounds of mortars going off, however, my body would have been where my arms were a second later. My luck held out that day.

I was sent back to get the scrape on my arm cleaned up and bandaged at the battalion aid station, located in a farmhouse just to our rear, leaving our own medic to take care of more urgent problems occurring on the line. Several wounded men were seated or stretched out on the ground in front of the house, probably having already been given some morphine to hold the pain. There were some very sad cases spread around the ground but the one that got to me was a GI sitting on a bench, holding up his injured hand, as if for his inspection. It was quite shredded, causing parts of it to hang down, making an extended extremity. He was crying, more in shock at what he was seeing than from pain, I believe. Maybe the thoughts of what such a wound would do to my amateur piano playing was at the bottom of it, but his condition bothered me more than some of the more seriously wounded. The grazing on my arm was so slight that its scar is barely discernible today, but it's one souvenir of the war that will always stay with me.

\section{Living with a Certain Numbness}
In war, there are always some deaths that occur accidentally, sometimes needlessly, and I observed a few of these events. In one case, our Captain, who was disliked by most of us, directed our Corporal, an American Indian, and a very good and experienced soldier, to scout out the area in front of us. The corporal objected as he was quite certain that there was a sniper in the forward woods. The Captain ordered him forward anyway\ldots\ He went, and he was killed by the sniper. Sometime later we advanced, passing his body, his helmet on his upright grounded rifle. Only rigid Army protocol protected the Captain from the resentment and anger we all felt.

It's worth noting that this Captain went entirely to pieces within a few days, having to be evacuated as a psycho case, which made none of us sorry. I didn't witness his disintegration, though I did in the case of a sergeant, who had been with the division right from the states and was a mighty tough guy in our estimation, became completely incapacitated, crying and refusing to get up out of his foxhole. Psychologists have written that all men are subject to being broken depending on their makeup and the experiences they have endured, which I truly believe and have witnessed. The infantry condition, with no policy of planned replacement of combat personnel, created a situation in which men came to believe that, eventually, they would either die or receive a serious injury, for that was the example they were witnessing as, in time, those around them were being steadily lost one way or another. Even the strongest came to live with a certain numbness, when such thoughts were pushed out of one's mind until the usual return of serious fear would grab hold once again.

\section{A Witness\ldots}
I witnessed another sad situation one day when we found ourselves in a fairly open area, being preceded up the hill behind a line of tanks. Many GI's were right behind the tanks as they made their slow progress toward the brow of the hill, while I was part of a group being kept back in reserve. Suddenly, the tanks and infantrymen were subjected to heavy incoming artillery, knocking down a lot of men. The tankers immediately hit the gas backing up, getting out of there as fast as possible. We witnessed the tanks running over some of the wounded GI's, counting at least seven men who were, for a certainty killed that way\ldots\ probably unavoidably, as the tankers must have been following hasty, possibly correct, orders by their leader.

\section{Mystery on the Horizon}
One of the weirdest sights I saw occurred one night when I was sitting in a foxhole, putting in my two hours of duty along a forest edge overlooking an open area. I was passing time by saying the rosary when suddenly some kind of projectile, not unlike a fourth of July rocket, came shooting across the distant sky parallel to the horizon. I might have concluded that it was a meteor except that suddenly it changed directions, aiming up into the heavens, where it finally disappeared from sight. A decade later, I might have called it a UFO, but the only conclusion I could make then was that it was one of the German's rocket-propelled weapons gone haywire. They were shooting rocket-propelled V-bombs to London at that time, but I wouldn't expect that Luxembourg would have been on the flight path. This will remain a mystery.

\section{The Smell of Death}
Another scene that has stuck in my mind occurred when, for some reason that I can't recall, we had reason to retrench our earlier steps, passing through a village that we had taken earlier in the day. The artillery and tankers had done quite a job in leveling most of the buildings which, in the dark of night, were still smoldering, some burning, with flames casting eerie shadows. Some poor cows were stumbling through the single village street, bellowing loudly in their need for milking and feeding. There were also several extremely bloated remains of other cows and horses that had been caught in the crossfire. Strangely enough, the harsh sounds of war seemed to be taking a break, creating a surprising quietness overall, with no small arms or shelling going on, just the crackling of the fires and the painful cries of the animals. Worst of all, however, was an overwhelming nauseating stench emanating from the destroyed village, a smell of charred remains of every sort, material and animal, which I can smell again at this moment. A smell of death, in a way, defining the desolate scene before us, a tribute to the futility of it all.

\section{The Incident at Dawn}
My only face to face hand to hand experience with the enemy in an area just beyond a major crossroads which our company had gained on the day, a significant accomplishment as one of the roads went to Bastogne, nine or ten miles away, which was surrounded and under heavy enemy pressure at that time. We got pinned down in the woods on the further side of the road, digging our foxholes as quickly as we could chip away at the frozen soil. Being unable to move around very much, I wound up with a different foxhole mate than Charlie Malasepena, my usual buddy from Brooklyn, with critical consequences occurring a few hours later.

After dark, I, along with two other guys, was dispatched to return to the building directly on the crossroads to recover some German sleeping bags, noticed earlier, which we coveted as further protection from the freezing weather. There were several bicycles in the yard and building, abandoned by the hastily departing Germans. We then encountered a severely wounded German soldier who was helpless groaning out something in German. We were of little use to him as the darkness precluded our passing him to the rear for treatment by our medics. It might have been a service to put him out of his misery but, for all we knew, his injuries may not have been life-threatening. He was disarmed, apparently by some of our people earlier in the day. If he survived he, presumably, would have been discovered by those behind us on the following day, given medical treatment and, most likely, have wound up eventually as a stateside POW, warm and comfortable, maybe helping to bring in some crop out west. We left him to whatever his fate was to be.

We were able to round up quite a few sleeping bags and, stealthily returning to our bivouac area, distributed them to our best friends, saving one for my new foxhole mate.

The usual pattern when holed up as we were, was for each man to take a two-hour stint of watchfulness while the other man slept. Just before dawn, I woke up, sitting up to take an  assessment of the situation. Instead of finding my ``buddy'' taking his turn at staying awake, I saw that he was sound asleep, completely zipped up in his German sleeping bag. I sat there debating whether to wake him up or let him sleep, as it was almost time for my turn at watching. At that point I sensed or heard some activity in the snow-laden forest in front of me, immediately seeing some vague forms moving through the snow-laden trees toward me in the dim early light. Upon my hasty shout of ``Halt!'' they continued to emerge toward me (I have never figured out why they didn't stop at my command). I fired and brought down the first German, while the second German, apparently concluding that I would fire on him next, ran forward and grabbed the end of my rifle, and we proceeded in a kind of tug of war for possession of the rifle. This development was made possible because I had not had time to check my rifle upon waking up, a routine task, and upon firing the first round my rifle jammed but good. This occurred because the accumulated snow had not been cleared from the track necessary for the automatic ejection of the first bullet and the advance of another one into the firing slot, with this snow being compressed into ice, putting the weapon out of action for the moment.

I was, of course, still within my sleeping bag from the waist down, wrestling for my rifle while my foxhole buddy was shouting and flubbing around trying to unzip his bag. As might be expected, I too was shouting for assistance which, fortunately, came from my true ``buddy'' Charlie, firing over our heads from his nearby foxhole. The other German immediately raised his hands in surrender and was taken prisoner. All of this occurred in far less time than is taken in this recounting. The fact that they had previously discarded their weapons made their intent known, but that was not evident to me as they came at me through the murky light of early dawn.

The wounded German was, along with his surrendering fellow soldier, sent back to our battalion area for treatment and processing. I hope they both survived and wonder how they might be describing this event today, an incident we shared in a snow-filled forest on that early Luxembourg dawn. I imagine that, as with me, it represents the most prominent memory of our total experience as combat infantrymen. Whatever happened later, that was their last day as fighting soldiers.

\section{Christmas Eve at War}
Christmas eve under war conditions must stick in every soldier's memory, with long thoughts about Christmases at home\ldots\ warn, happy, going to Midnight Mass. The \nth{24} of December found us in Kotschette, a crossroads town which we, after a few skirmishes, cleared of Germans by late afternoon. Some of us were distributed to dig in, forming a protective line outside of the crossroads, while others, including me, were fortunate to be allowed to spend the night in a small inn called, as I learned upon returning years later, Hotel Weis-Welter.

Two events occurred shortly upon entering the building. Two Germans wearing white camouflage clothing were spied working their way toward us in a nearby field, quickly removed from action by those closest to them outside. After that we started to enjoy our good fortune, opening some of the wine bottles and, in my case, starting to pick out a tune on a piano found in the main room of the inn. Suddenly there was a loud explosion upstairs, where other guys were scouting out the bedrooms. It turned out that one of the men had failed to return the safety pin to his unused rifle grenade (unlike a hand-thrown grenade, this one is fired from the end of a rifle, requiring the removal of the pin before firing, but not activated until fired). Somehow, he had accidentally triggered his rifle. All activity, especially the piano playing, came to a halt as we first thought it might have been a German booby trap\ldots\ A piano would have presented that possibility. The four men in that upstairs room were brought out and laid on the ground in front of the inn, getting serviced by the medics. As I recall, I believe only two of them lived.

My sleeping spot for the night was on the floor of the main room, and it so happened that I was on guard duty at the front door of the inn when Christmas commenced. Possibly out of mutual respect for the holiday, there was no artillery, mortar, or any kind of fire going on that night. While my guarding stint was uneventful, I've always recalled that Christmas Eve at some point during the many happy ones that have followed.

The events on Christmas Day were described in a Christmas letter that I wrote several years ago, which will be enclosed as an appendix.

\section{Piglet for Dinner}
One pleasant memory, for a change of pace, was that of a delicious piglet dinner we had one night. We were spending the night in a farmhouse and one of our men, experienced in such things, killed a piglet and then cooked it in a frying pan on the wood-fired stove. How succulent it was!

\section{A Night of Pain}
My days in combat came to a close sometime in mid-January (wish I had been able to keep a journal). At this point, our company membership had, through combat losses, declined so drastically that we were combined with other remnants from other units to make up a reasonable combat group. It developed one evening that, due to my general declining physical condition, I was allowed, along with others in similar shape to spend the night in a farmhouse well behind our line of combat. Surprisingly, the farmer and his wife were still in the place, and they heated some food, which we ate in front of a roaring open fireplace. After dinner, the farmer started playing a few tunes on an accordion, making for a wonderful respite from the preceding days.

That became, for me, a night of horrible pain. I had taken off my boots to dry my socks in front of the fire, and I went to sleep with my right side close to the hearth. Within a few hours my unshorn feet, particularly my right foot, began to swell up, eventually, the right one looked like it could burst, it was so distorted, and the pain was excruciating. By morning I was incapable of walking\ldots\ my feet were useless\ldots\ so large that I could not put my overshoes over them, never mind my combat boots. I, along with another guy with the same problem, trench foot, was placed on a stretcher mounted on a jeep and brought back to the battalion aid station, then going by military ambulance to a tented field hospital, where I spent the night. The pleasure of sleeping on a cot and being cared for by genuine American nurses was offset by the serious condition of some of the men around me. The man closest to me, still under the effect of the anesthesia after having his leg removed, started flirting with the nurse, then burst into a happy song, yet to learn of his loss, his joy turned to sobbing later in the night.

The next day we were brought by ambulance to a general hospital in Paris, en-route to what would become my months of recovery in England. My combat days had come to an end truth be told, that did not make me unhappy. Except for leaving Charlie Malasapena, I had little of loyalty to what remained of my company, glad to leave before my number might be called in a more serious way, which I saw as an inevitable outcome in one way or another in view of what was going on at that time.
\end{document}
