\documentclass[../m3y]{subfiles}
\externaldocument{../m3y}
% \graphicspath{{\subfix{../images/}}}
\begin{document}
It is likely that, just as in all endeavors, one's experiences and their consequences upon the body and mind of the individual affected may differ considerably from others in similar circumstances. Despite that fact, my experiences and the effect they had upon me probably are not very distant from those who were in the same spot and time as I.

The overarching concern for all of us was that we now found ourselves naked, in the sense that there was nothing between us and those who would kill us, sometimes being close enough to hear shouted commands or whatever between the Germans. This put us on the receiving end of whatever was thrown at us\ldots\ rifle and machine-gun fire, mortars, powerful artillery, tanks, and the dreaded Sch{\"u}-mine, a vicious booby trap which would only blow off only your foot if you were lucky. Such kind of close-up action never seemed to be one-sided, rather a constant pounding exchange of terror between the facing forces.

It's not shameful to admit that fear of being hurt or killed was a constant worry under such conditions, and I was not spared those concerns. One author has stated that ``Physical courage is little more than the ability to control the physical fear which all normal men have, and cowardice does not consist in being afraid but in giving away to that fear.'' Granting that, one might ask how the individual manages to overcome and control this understandable fear? It lies somewhere in one's desire to retain the good opinion of one's fellow with pride helping to smother fear somewhat. Another compelling factor is that each Gl knew that he was as dependent upon the performance of the other guy as he (the other guy) was dependent upon you. Sociologists might classify it as a ``social exchange.'' Whatever\ldots\@. We needed each other and you got your share of another's caring by displaying the same interest in his welfare. Though we were strangers to each other just days before, certain alliances were formed quickly, particularly as we would buddy up, two men to a foxhole, sharing the whole extreme business together. Mutual misery can bring about much that is good as well as that which is less admirable with some individuals. Certain guys, some truly slobs, were to be avoided, while, as happened to me at one memorable moment, another man might save your life, as I shall describe at another point.

The other constant remembered most vividly, and a concomitant threat to all in Luxembourg at that time was the weather… fearful, unremitting cold, heavy snows, and an all-around miserable physical environment. One soon experienced numbness in all extremities, it becoming an abiding condition of daily existence. Those particular days have been described as the worst winter weather experienced in central Europe in the entire century, which will get no argument from me. It bears mentioning that a combat rifleman spends, with rare exception, all 24 hours outside, most of it in a miserable foxhole dug by him and his foxhole buddy, often snuggling together to retain whatever warmth might be available, often going days on end in that situation. This is what ultimately ended my combat experience, as my family is well aware, putting me in the hospital for close to five months as a result of severe trench foot. More about this later.

Much of my life at that time seems a bit of a jumble, in that I have difficulty in reconstructing subsequent days with precision. Certain events remain, however, some extremely vivid, and they will have to serve as a few examples of the elements comprising my brief but intense life as a combat rifleman. These, recounted in no particular order, some of lesser or greater importance, but not without interest, are what follows. That they are my most vivid memories accounts for their egregious aspects.
\end{document}
